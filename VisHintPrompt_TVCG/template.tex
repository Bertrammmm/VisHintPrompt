\documentclass[lettersize,journal]{IEEEtran}
\usepackage{amsmath,amsfonts}
\usepackage{algorithmic}
\usepackage{algorithm}
\usepackage{array}
\usepackage[caption=false,font=normalsize,labelfont=sf,textfont=sf]{subfig}
\usepackage{textcomp}
\usepackage{stfloats}
\usepackage{url}
\usepackage{verbatim}
\usepackage{graphicx}
\usepackage{cite}
\hyphenation{op-tical net-works semi-conduc-tor IEEE-Xplore}
% updated with editorial comments 8/9/2021
% % 额外包
\usepackage{multirow}   % 跨行
\usepackage{graphicx}   % 调整表格大小
\usepackage{booktabs}      % 专业三线表
\usepackage{array}         % 支持 m{宽度} 列
\usepackage{makecell}      % 表头换行
\usepackage{siunitx}       % 数字对齐
\usepackage{adjustbox}     % 调整表格到页宽
\usepackage{siunitx}  % for S column type alignment
\usepackage[table]{xcolor} % 在导言区引入
\usepackage{amsmath}
\usepackage{float}
\usepackage{colortbl}
\usepackage{xcolor}
\usepackage{diagbox}
\usepackage{hyperref} 
\usepackage{cleveref}

% \usepackage{amsmath,amssymb}
% \usepackage{algorithm}
% \usepackage{algpseudocode} % algorithmicx
% \algrenewcommand\algorithmicrequire{\textbf{Input:}}
% \algrenewcommand\algorithmicensure{\textbf{Output:}}
% \algrenewcommand\algorithmiccomment[1]{\hfill\(\triangleright\)~#1}

% 定义一个宏,用于自动给第二列及后续列加灰底
\newcommand{\visrow}[2]{%
  & \cellcolor{gray!15}#1 & \cellcolor{gray!15}#2 \\
}
\newcommand{\toolName}[1]{\textit{VisHintPrompt}}

% % siunitx 设置
\sisetup{
  table-number-alignment=center,
  table-format=2.2,  % 两位整数+两位小数
  detect-all
}

\usepackage{pifont} % for check marks
\newcommand{\cmark}{\ding{51}} % ✓
\newcommand{\xmark}{\ding{55}} % ✗

%% We encourage the use of mathptmx for consistent usage of times font
%% throughout the proceedings. However, if you encounter conflicts
%% with other math-related packages, you may want to disable it.
\usepackage{mathptmx}                  % use matching math font
\usepackage{soul}
\newcommand{\yumeng}[1]{
    \textbf{\color{purple}[yumeng: #1]}
}
\newcommand{\wy}[1]{\textcolor{brown}{[Yong: #1]}}

\newcommand{\pyl}[1]{
    \textbf{\color{blue}[Peng: #1]}
}


\begin{document}

\title{VisHintPrompt: A Visual Hint Prompting Strategy for Numerical Inference from Charts}

\author{
Fengling Zheng, Yongle Peng, Zi Rong, Chenyun Cai, Yumeng He, Dekun Qian,\\
Zhiguang Zhou, Yigang Wang, and Yong Wang% <-this % stops a space
\thanks{Fengling Zheng is with Hangzhou Dianzi University and Nanyang Technological University. A part of this work was done when she was a visiting student supervised by Yong Wang at Nanyang Technological University. E-mail: \{fenglingzheng\}@hdu.edu.cn.}%
\thanks{Yongle Peng, Zi Rong, Chenyun Cai, Dekun Qian, Zhiguang Zhou, and Yigang Wang are with Hangzhou Dianzi University, Hangzhou, China. E-mail: \{22330236, rongzizi, 23280129, qiandekun, zhgzhou, yigang.wang\}@hdu.edu.cn.}%
\thanks{Yumeng He is with the Viterbi School of Engineering, University of Southern California, United States of America. E-mail: heyumeng@usc.edu.}%
\thanks{Yong Wang is with Nanyang Technological University, Singapore. E-mail: yong-wang@ntu.edu.sg.}%
\thanks{(Corresponding authors: Yong Wang and Zhiguang Zhou.)}%
}

% \author{IEEE Publication Technology,~\IEEEmembership{Staff,~IEEE,}
%         % <-this % stops a space
% \thanks{This paper was produced by the IEEE Publication Technology Group. They are in Piscataway, NJ.}% <-this % stops a space
% \thanks{Manuscript received April 19, 2021; revised August 16, 2021.}}

% The paper headers
\markboth{Journal of \LaTeX\ Class Files,~Vol.~14, No.~8, August~2021}%
{Shell \MakeLowercase{\textit{et al.}}: A Sample Article Using IEEEtran.cls for IEEE Journals}

% \IEEEpubid{0000--0000/00\$00.00~\copyright~2021 IEEE}
% Remember, if you use this you must call \IEEEpubidadjcol in the second
% column for its text to clear the IEEEpubid mark.

\maketitle



% A teaser figure can be included as follows
% \begin{figure*}[t]
%   \centering
%   \includegraphics[width=\linewidth]{figs/teaser.png}
%   \caption{
%    Overview of \toolName{}, a visual hint prompt engineering framework for data inference from charts.
%   The method operates in three stages:
%   (1) \emph{axis-aware grid enhancement}, which exposes fine-grained axis priors by overlaying grids aligned with the chart’s underlying structure;
%   (2) \emph{visual feedback prompting}, which introduces model-informed alignment cues to refine intermediate predictions; and
%   (3) \emph{progressive zoom-in refinement}, which enlarges the relevant region to provide high-resolution visual detail for fine-grained and reliable value inference across both Cartesian and radial charts.
%   }
%   \label{teaser}
% \end{figure*}

%% Abstract section.
\begin{abstract}
Numerical inference from charts underpins many visualization tasks, including chart question answering and chart redesign. Although many chart-specific models have been developed for numerical extraction, their reliance on the chart types and visual encodings represented in training data limits their 
% generalization. 
generalizability.
Modern Multimodal Large Language Models (MLLMs) possess competitive general-purpose visual understanding, offering the potential for numerical inference without additional 
% raining. 
training.
Yet how to reliably elicit such fine-grained numerical inference from pretrained MLLMs via visual prompting remains largely unexplored. To address this gap, we propose \toolName{}, a scaffolded visual hint prompting strategy grounded in the explicit extraction of axis priors from both Cartesian and polar charts. \toolName{} elicits the latent numerical inference abilities of MLLMs by progressively narrowing the search region. It supplies structured visual cues that guide the model toward accurate quantitative interpretation. The strategy consists of three coordinated components: \emph{Axis-aware Grid Enhancement}, which 
% exposes 
extracts
fine-grained axis priors and further exposes them as structure-aligned grids; \emph{Iterative Visual Feedback}, which overlays
% model-informed 
alignment cues derived from intermediate predictions to enhance the next prediction; and \emph{Progressive Zoom-in Refinement}, which enlarges the region of interest and applies progressive
grid densification to enhance local detail for precise value estimation. \toolName{} is visual-centric and applies to both Cartesian charts (e.g., bar, scatterplot) and polar charts (e.g., pie, radar).
Experiments across diverse chart types on both proprietary and open-source MLLMs demonstrate consistent improvements over standard prompting. Quantitative and qualitative analyses, along with ablation studies, validate the effectiveness and essential role of each component. These results underscore the broad applicability of \toolName{} in enabling reliable and more accurate numerical inference from charts.
\end{abstract}
\begin{IEEEkeywords}
Chart understanding, multimodal large language models, visual prompting, data inference, and information visualization.
\end{IEEEkeywords}



% \section{Introduction}

\section{Introduction}
\IEEEPARstart{N}{umerical} inference from charts is a fundamental capability that can benefit a wide range of visualization tasks, including chart question answering~\cite{Deplot,ChartQA,PlotQA} and chart redesign~\cite{ReVision}. It requires accurately mapping visual encodings, such as positions, lengths, or angles, to their corresponding numerical values. Prior research primarily addressed chart numerical inference through chart-specific models designed for particular chart types or visual encodings~\cite{ChartOCR,UniChart,cuarbune2024chart}. While these models can achieve competitive performance within their targeted settings, their effectiveness often relies on assumptions about chart structure and encoding styles, limiting their ability to generalize to unseen or structurally diverse charts.

Recent advances in Multimodal Large Language Models (MLLMs) have 
% attracted significant attention due to their 
shown
strong general-purpose visual and linguistic understanding capabilities,
which motivates growing efforts to adapt MLLMs to chart-related numerical inference tasks through task-specific training, including specialized dataset construction and fine-tuning or instruction tuning~\cite{visrefinstuning,UniChart}. Representative approaches include Deplot~\cite{Deplot}, and ChartSketcher~\cite{ChartSketcher}. While these methods have shown promising performances on chart understanding benchmarks, they rely on substantial task-specific data and training to modify model behavior, often incurring high computational and data collection costs.

An alternative and potentially more lightweight approach is to enhance MLLMs’ chart numerical inference through visual prompting. A recent evaluation study, ChartInsights~\cite{ChartInsights}demonstrates that explicitly augmenting charts with visual cues can significantly enhance models’ ability to analyze data and perform reasoning, without additional training. Beyond chart-specific tasks, a growing body of work in general computer vision has explored visual prompting strategies for object detection, keypoint localization, and spatial reasoning. Methods such as DetToolChain~\cite{DetToolChain}, RedCircle~\cite{shtedritski2023does}, and SCAFFOLD~\cite{lei-etal-2025-scaffolding} provide converging evidence that explicitly externalizing spatial priors through image manipulation, such as zoom-in operations and coordinate-reading cues (e.g., overlaying rulers, compasses, or dot matrices), can effectively guide MLLMs’ attention and improve spatial localization.

However, accurate chart numerical inference fundamentally requires establishing explicit correspondences between visual coordinates and numerical values under axis constraints. 
While existing visual prompting strategies have shown effectiveness in extracting explicitly annotated values and improving spatial localization, it remains unclear whether and how visual prompting alone can systematically enable pretrained MLLMs to perform fine-grained numerical inference, where values must be inferred from visual encodings via precise alignment with axis scales. 
This unresolved question motivates the need for a systematic and generic visual prompting strategy for chart numerical inference. 


To fill this gap, we propose \toolName{}, a scaffolded visual prompting strategy designed to support fine-grained numerical inference from charts using pretrained MLLMs, without requiring any additional training. The core idea of \toolName{} is to explicitly expose axis-aware spatial structure by introducing and progressively refining grid-based visual references, which serve as a foundation for structured visual hints to guide a more accurate numerical inference (Fig.~\ref{fig:IntroductionFig}). 
% In particular, we
Our experiments
show that explicitly extracting and externalizing \emph{axis priors}, i.e., axis-related structural information such as axis orientation, tick locations, and value scales, provides an effective geometric foundation for guiding fine-grained numerical inference.
% Grounded in the explicit extraction of axis priors from both Cartesian and polar charts using simple and off-the-shelf image processing algorithms, \toolName{} integrates structured visual hints with multi-round refinement to progressively narrow the effective search space and strengthen spatial grounding during inference. 
Specifically, \toolName{} consists of three coordinated components: \emph{(1) Axis-Aware Grid Enhancement}, which introduces structure-aligned Cartesian or polar grids to provide a stable reference for mapping visual positions to numerical values; \emph{(2) Iterative Visual Feedback}, which overlays alignment cues derived from intermediate predictions to iteratively correct coarse localization errors; and \emph{(3) Progressive Zoom-in Refinement}, which selectively enlarges regions of interest and applies progressive grid densification to introduce finer axis-aligned references, amplifying subtle mark-axis relationships that are critical for precise value estimation. 
These components form a coherent strategy that incrementally refines spatial focus and numerical grounding.
Experiments across diverse chart types on both proprietary and open-source MLLMs show consistent performance gains over standard prompting. Quantitative and qualitative analyses, together with component-wise ablation studies, further validate the complementary roles of each component. Our main contributions are summarized as follows:


\begin{itemize}
    
    \item \textbf{A scaffolded visual hint prompting strategy (\toolName{}).}
    We propose \toolName{}, a structured visual prompting framework that integrates Axis-aware Grid Enhancement, Iterative Visual Feedback, and Progressive Zoom-in Refinement. By progressively narrowing the model’s effective search space, \toolName{} enables more precise numerical inference from charts with pretrained MLLMs, without additional training.
    
    \item \textbf{A comprehensive evaluation of \toolName{} on numerical inference across different MLLMs and chart types.}
    We conduct systematic experiments spanning three proprietary and two open-source MLLMs across nine chart types, covering both Cartesian and polar encodings. Our evaluation integrates quantitative results, qualitative analysis, and component-wise ablation studies. The results demonstrate consistent improvements over standard prompting and confirm the role of each component.

\end{itemize}

\begin{figure}[tb]
  \centering
  \includegraphics[width=\columnwidth, alt={Examples of scatterplot data inference with an MLLM.}]{figs/Introduction.png}
  \caption{%
  	An illustrative example of scatterplot data inference using an MLLM with or without visual hints (i.e., the grid here). Black text represents the ground truth, red text highlights large numerical extraction errors, and green text indicates accurate extraction with minimal deviation from the ground truth. Grid hints substantially improve extraction performance.    
  }
  \label{fig:IntroductionFig}
\end{figure}

\begin{figure*}[t]
  \centering
  \includegraphics[width=\linewidth]{figs/teaser.png}
  \caption{
   An overview of \toolName{}, a visual hint prompting strategy for numerical inference from charts.
  The method operates in three components:
  (1) \emph{Axis-aware Grid Enhancement}, which exposes fine-grained axis priors by overlaying grids aligned with the chart’s underlying structure;
  (2) \emph{Iterative Visual Feedback}, which introduces model-informed alignment cues to refine intermediate predictions; and
  (3) \emph{Progressive Zoom-in Refinement}, which enlarges the relevant region to provide high-resolution visual detail for fine-grained and reliable numerical inference across both Cartesian and polar charts.
  }
  \label{teaser}
\end{figure*}



\section{Related Work}

In this section, we review three lines of research most relevant to our work: (1) traditional chart information extraction approaches, and
% that reconstruct structured data from chart images;
(2) chart understanding enhancement strategies for MLLMs from two perspectives: model-centric approaches that enhance chart understanding via pretraining or instruction tuning and visual-centric prompting strategies that guide MLLMs without changing model parameters.


\subsection{Traditional Chart Information Extraction Approaches}
For decades, researchers have devoted efforts to extracting structural information from chart images to recover their underlying structured data, thereby supporting downstream tasks such as chart understanding and analysis. Early approaches typically adopted multi-stage parsing pipelines. The core idea was to first detect chart elements using image processing and OCR techniques, and then reconstruct values based on geometric or semantic relationships. Since different chart types are defined by distinct graphical primitives, these methods often relied on predefined rules to locate components through color continuity and edge features, followed by graphical mark extraction\cite{ReVision}. For example, \textit{ReVision}\cite{ReVision} employed edge detection and pattern recognition techniques to separate axes, data marks, and legends, thereby recovering data from bar and pie chart images. \textit{ChartSense}\cite{ChartSense} leveraged a mixed-initiative design that combined image processing with guided user interaction to achieve fast and accurate data extraction. \textit{ChartKG}\cite{ChartKG} integrated object recognition, OCR, and rule-based parsing of graphical marks, and organized the recovered data in the form of a knowledge graph. Although such approaches provide interpretability, they depend on complicated pipelines, where errors can be amplified across multiple stages, ultimately limiting robustness. With the advent of deep learning, research has shifted toward end-to-end frameworks for chart data extraction, aiming to directly predict structured outputs from chart images. \textit{Scatteract}\cite{Scatteract} automatically recovered data points from scatter plots. \textit{DVQA}\cite{DVQA} proposed a deep dual-network model to parse values directly from bar chart images. \textit{ChartOCR}\cite{ChartOCR} combined deep learning with rule-based methods, extracting key points of chart elements to support data extraction from three common chart types. However, these algorithms were often tailored to specific chart types and only covered a limited set of formats (e.g., line, bar, and pie charts), showing clear limitations in handling the diversity and complexity of real-world charts. More recent studies have introduced end-to-end Vision–Language Models\cite{Pix2Struct, MatCha, UniChart, ChartT5}, which learn mappings from chart images to structured tables via vision encoders and autoregressive decoders. By leveraging weak supervision or self-supervision for cross-modal alignment, these methods aim to achieve direct chart understanding from images. However, in the absence of explicit numeric annotations or under diverse chart styles, their ability to extract visual structural information and to generalize remains limited, which will be addressed in this work.


% \subsection{Enhancing MLLMs for Chart Understanding}
\subsection{Chart Understanding Enhancement Strategies for MLLMs}
% \wy{Fengling, pls check if the change is aligned with your intent, and revise the paragraphs below.}

The emergence of MLLMs such as GPT-4o, Gemini, and LLaVA has further transformed the paradigm of chart understanding, with end-to-end reasoning becoming increasingly mainstream. A variety of strategies have been proposed to improve their performance in chart data inference, which can be broadly categorized into \textit{model-centric enhancements} and \textit{visual-centric prompting strategies}.

% \wy{For the following paragraphs, 1) the introduction of existing work can be condensed; 2) the difference between prior work and our approach is a bit redundant and can be shortened.}
\subsubsection{\textbf{Model-centric Enhancements}}

Model-centric approaches improve chart understanding primarily through large-scale pretraining, instruction tuning, or architectural adaptation using chart-specific data~\cite{CharXiv,NLDataset,NovaChart,ChartInstruct,QwenChart}. Representative methods fine-tune or pretrain MLLMs on synthetic or curated chart–table pairs to align visual and tabular modalities~\cite{han2023chartllama,ChartAssistant,MMC}, or construct high-quality instruction datasets with diverse visual details to strengthen fine-grained perception~\cite{EffectTrainingData}. Other studies explore efficiency-oriented or structural enhancements. TinyChart~\cite{TinyChart} introduces a lightweight model and a parameter-free visual instruction merging module to reduce the computational burden of high-resolution inputs. Visualization-referenced instruction tuning~\cite{visrefinstuning} enhances fine-grained perception by partially unfreezing vision encoders and adopting hybrid-resolution strategies. OneChart~\cite{Onechart} incorporates auxiliary tokens to guide attention toward critical chart components, while CHOPINLLM~\cite{PretrainingMLLM} integrates original data alignment during pretraining and explicitly extracts chart data before question answering. ChartGemma~\cite{ChartGemma} trains directly on instruction data generated from chart images without relying on underlying data tables, and ChartMoE~\cite{ChartMoE} leverages a mixture-of-experts architecture to improve multi-task alignment. Chart-r1~\cite{Chart-r1} calibrates numerical sensitivity via reinforcement fine-tuning.
 
While these approaches have achieved remarkable advances, they either incur high training and deployment costs or are intrinsically proprietary and impossible to conduct training.



\subsubsection{\textbf{Visual-centric Prompting Strategies}}

Visual-centric prompting strategies enhance chart understanding by injecting structured visual cues into the input and guiding MLLMs to perform step-by-step reasoning without modifying model parameters~\cite{DetToolChain}. Early work such as ChartThinker~\cite{ChartThinker} introduced chain-of-thought reasoning for charts, but primarily focused on textual decomposition rather than explicit visual parsing. Subsequent approaches increasingly integrate visual guidance to complement textual reasoning. ChartInsights~\cite{ChartInsights} and VisualCoT~\cite{VisualCoT} employ region-focused visual cues and cropping strategies to improve local visual grounding. DetToolChain~\cite{DetToolChain} further introduces detection-based prompting with staged visual hints, coordinate measurement, and self-verification to enhance consistency across iterations. VProChart~\cite{VProChart} models human-inspired visual alignment principles, such as spatial proximity and crosshair alignment, to significantly improve numerical inference accuracy.

These methods show that localized visual cues can effectively support chart reasoning, but they typically focus on isolated regions or task-specific scenes. 
In contrast, our approach introduces an axis-aware visual scaffold that enables coarse-to-fine numerical inference by combining global structural context with localized refinement.


\section{VisHintPrompt}
% Grounded in explicit extraction of axis priors from both Cartesian and polar charts, 
We propose \toolName{}, a visual-hint-based prompting strategy to elicit MLLMs’ abilities of numerical inference from charts. It introduces three coordinated visual hint prompting components (Fig.~\ref{teaser}(b)), \emph{(1) Axis-aware Grid Enhancement}, \emph{(2) Iterative Visual Feedback}, and \emph{(3) Progressive Zoom-in Refinement}, which progressively narrow the search region and substantially improve numerical accuracy and robustness across diverse chart types. \toolName{} begins with chart type classification, which determines the appropriate axis prior extraction procedure and the corresponding visual prompting strategy. 
We consider two coordinate families: Cartesian (bar, line, scatterplot, and bubble charts) and polar (pie, donut, radar, and rose charts).

% This section first describes the three visual hint prompting components of \toolName{}, and then summarizes the overall prompting pipeline used in our experiments. We consider two coordinate families: Cartesian (bar, line, scatter, and bubble charts) and polar (pie, donut, radar, and rose charts).

% While modern MLLMs such as GPT-4o and Gemini achieve notable success in reasoning and recognition, accurate numerical extraction from charts continues to be an underexplored capability. To unleash the potential of MLLMs in this regard, we propose a comprehensive visual prompting strategy, VisHintPrompt, whose workflow is illustrated in \cref{fig:teaser}. A toolkit is devised to assist image processing, including Axis \& Tick Detector, Grid Generator, Visual Feedback Iterator, and Regional Amplifier. In addition, chain-of-thought textual prompts are designed to guide the MLLM in executing the VisHintPrompt pipeline, thereby accomplishing data inference from chart images.
\subsection{Axis-aware Grid Enhancement}
\subsubsection{\textbf{Extraction of Axis Priors from Chart Images}}
Axis-prior extraction (Fig.~\ref{teaser}(a)) aims to obtain high-precision axis information that is readily available from chart images using simple, off-the-shelf image processing algorithms. 
% In our setting, Cartesian charts include bar, line, scatter, and bubble charts, whereas polar charts refer to plots whose data positions are encoded by angle and/or radius around a common center, such as pie charts, donut charts, rose charts, and radar charts. 
Specifically, the extracted axis priors across these two coordinate systems comprise the geometric layout (x–y axes or circular frame, axis positions, circle center, and radii) and the associated numeric structure (tick locations, tick labels, and scales), providing a high-precision reference for subsequent visual-hint prompting.

% --------------------- Cartesian ---------------------
\paragraph{Axis priors in Cartesian coordinates}
For Cartesian charts, we extract axis priors describing the x–y axis geometry and numeric scale.

\textbf{Extraction of x–y axes.} We begin by detecting candidate horizontal and vertical axis lines using the classical Hough Transform, which is robust for identifying long linear geometric structures. Among the detected lines, we select the pair that satisfies the characteristic property of Cartesian coordinates—one dominant horizontal line and one dominant vertical line that are orthogonal to each other. To additionally ensure robustness across charts with different axis placements, we employ an MLLM to identify the layout of the chart and verify the correct axis pair. This combination of geometric detection and layout validation reliably yields the final axis extraction for Cartesian charts.

\textbf{Alignment of tick values to pixel positions.} Tick marks are first scanned along each extracted axis as short line segments that are perpendicular to the axis line. Because this scanning process may occasionally be affected by background noise or decorative elements, we analyze the distribution of intervals between adjacent detected ticks and identify the interval that appears most frequently as the true tick spacing. Using this dominant spacing, we refine the tick locations and correct potential outliers, which produces a more reliable set of tick positions. 
% For each refined tick, we apply OCR\pyl{OCR is not utilized in the Cartesian coordinate system currently} to obtain the corresponding numeric label and pair it with its pixel coordinate. 
Through this refinement procedure, the method produces accurate pixel–value pairs that support stable and precise value interpolation in subsequent components of our prompting framework.

% --------------------- polar ---------------------
\paragraph{Axis priors in polar coordinates}
For polar charts, we extract axis priors that capture both the polar geometric layout and the two forms of numeric encoding in polar coordinates: angular encoding in pie and donut charts, and the joint angular–polar encoding used in radar charts.

\textbf{Extraction of circle center and radius.}
To obtain the geometric parameters of polar charts, we detect the chart center and its outermost circular boundary. The image is converted to grayscale, smoothed with a Gaussian filter, and processed with a Canny operator to enhance structural edges. The resulting edge map is analyzed using the Hough Circle Transform (CHT), which is robust to noise and partial edge loss and returns a set of candidate circles. Because polar charts often contain multiple concentric rings, we select the circle with the largest radius and adopt its center $(x_c, y_c)$ and radius $r_{\text{max}}$ as the geometric parameters of the chart.

\textbf{Alignment of polar values to pixel radii.}
For pie and donut charts, the polar coordinate structure is fully determined by the center and the outer boundary, since these charts encode values solely through angular spans. In contrast, radar charts and rose diagrams encode data through polar distances, which requires an additional alignment between pixel radii and numeric values. To obtain this polar mapping, we detect two reference circles using CHT and extract the annular region between their radii $r_1$ and $r_2$. This region contains the numeric tick labels corresponding to the two grid circles. The annular patch is provided to the MLLM, which reads the labels and yields the paired observations $(r_1, V_1)$ and $(r_2, V_2)$. With the shared center $(x_c, y_c)$ and these two radius--value pairs, we construct a linear mapping from pixel radius to numeric values. This mapping defines the chart’s radial metric and completes the polar-axis representation for subsequent prompting components.

% --------------------- Unified ---------------------
\paragraph{Unified axis-prior representation}
% To support a unified prompting design, we store the extracted axis priors for both Cartesian and polar charts in a common representation. This representation exposes, for each chart, the valid value ranges, orientation of each axis (linear or polar), and the corresponding pixel-to-value mappings. In the following subsections, we use these axis priors to construct visual hints for grid prompting, prediction overlays, and progressive zoom-in refinement in \toolName{}.
To enable a unified prompting design, we store the extracted axis priors for Cartesian and polar charts in a shared representation that encodes valid value ranges, axis orientation, and pixel-to-value mappings. These priors are then used to construct grid-based visual hints, visual feedback
% \pyl{The terminology is not aligned with the previous context.}
, and progressive zoom-in refinement in \toolName{}.

\subsubsection{\textbf{Geometry-aligned Grid Construction}}
% \pyl{It shares the same name as the main heading.}
Building on the extracted axis priors, this component overlays geometry-aligned reference grids as explicit visual cues for MLLMs. Depending on the chart type, these grids take the form of orthogonal grids, concentric rings, or angular polar divisions, corresponding to different geometric encoding families 
% (Fig.~\ref{teaser}(b))
. 
By embedding axis-guided visual scaffolds into the chart, the perceptual distance between data marks and nearby reference boundaries is reduced, enabling value inference within a more localized and interpretable neighborhood.
% \pyl{In my opinion, a discussion integrating global perspectives and region-specific insights of interest could be added to this section, so as to address Yong’s question raised earlier: "Are our approaches also for local regions?"}

% Building on the extracted axis priors, this component constructs a set of 
% geometry-aligned reference grids that serve as explicit visual cues for 
% multimodal LLMs. These reference structures include orthogonal grids, concentric 
% rings, and angular polar divisions, each corresponding to a distinct geometric 
% encoding family and illustrated in Fig.~\ref{teaser}(b). By embedding such axis-guided visual 
% scaffolds directly into the chart, the perceptual distance between data marks and 
% their nearest reference boundaries is substantially reduced, enabling the model to 
% perform value inference within a narrower, more interpretable local neighborhood.

\paragraph{Orthogonal grid hint}
This form of prompting applies to charts embedded in Cartesian coordinates, such as 
bar charts, line charts, scatterplots, and bubble charts. For value-encoded axes, 
we generate a fine-grained orthogonal grid by subdividing the original tick 
intervals into visually manageable spans.
% , approximately 50 pixels in width, as shown in Fig.~\ref{fig:case1-1}(Scatter)\pyl{The figure is excessively distant from the corresponding text.}
These subdivisions maintain integer tick values whenever 
possible to avoid excessive floating-point annotations. For categorical axes, only 
label-aligned guide lines are inserted. Scatterplots and bubble charts adopt the 
same subdivision strategy symmetrically along both dimensions to preserve the 
underlying value mapping.

\paragraph{Concentric ring hint}
Charts such as radar and rose plots encode data magnitudes along polar directions. 
For these chart families, we construct a hierarchy of concentric rings by 
subdividing the polar axis, thereby generating finer reference levels around the 
original polar ticks.
% , as illustrated in Fig.~\ref{fig:case}(Rose)\pyl{The figure is excessively distant from the corresponding text.}. 
These rings preserve the polar 
geometry while supplying clearer reference distances for the model to interpret 
polar magnitudes.

\paragraph{Polar-division hint}
Donut and pie charts encode proportional information through angular spans. For 
these charts, we construct polar-division prompting by inserting angular 
divisions aligned with the chart’s outer radius, as shown in Fig.~\ref{fig:case1} (Donut). These 
divisions are optionally annotated with angle values, enabling the model to directly 
read start and end angles for proportion estimation.
% in a geometrically explicit manner.

Together, these geometry-specific prompting structures form a unified 
axis-guided reference framework that extends across Cartesian and polar chart 
families. By providing clear, geometry-aligned visual anchors, this component 
substantially enhances the model’s ability to localize, interpret, and quantify 
underlying chart encodings.

% \subsubsection{Axis-Guided Grid Prompting}
% Build on the axis priors extraction, Axis-Guided Grid Prompting aims to provide more explicit reference with extra or more fine-grained localization axis information by overlaying lines, polars and arcs with tick value on the original chart image, as depicted in Fig. 3. The auxiliary grid hints enable MLLMs to output accurate data values with the help of more nearest/direct accessible references overlaid in the chart image. This tick-related marks minum the distance data marke and reference tick line, allowing MLLMs to in a more little span to predict the accurate data values. Across charts with different axis style, the construction of specific axis-guided grid is detailed as follows:\\
% Cartesian charts here includes bar line scatterplot bubble charts. Divided by the dimension of data mapping, bar/line charts only mapping data in one value axis, for that the grid hint is emphasized to the value axis. As shown in fig.3(a), to ensure the more fine-grained grids and the tick values visible not overlapping, we set the grid span to be around 50 pixel. In addition, the span is set by dividing the original span, to keep the fine-grained  grid with int tick value, avoid too many float tick value annotation. For class axis, the lines only need to render from the label. For the scatterplot and bubble charts, the two value axis follow the same rule to curate expected grid hints.\\
% To construct the grid prompting for polar(ploar) charts, there is different from the donut chart and pie chart to radar chart and rose chart due to their different data mapping, as shown in fig.3(b). For the chart which mapping propotion data, such as donut and pie chart, we construct a circular lines that align with polar direction to enable MLLMs read the start and end angles directly to compute the data value. Different from this, radar and rose chart is mapping data with polar, in which we follow the original arc axis to fine-grained the tick lines, as shown in fig.3(c).

\begin{figure}[b]
  \centering
  \includegraphics[width=\columnwidth, alt={Examples of scatterplot data inference with an MLLM.}]{figs/grid-refine.png}
  \caption{
  % \pyl{The vertical line segment of the solid line for R3 is missing in this plot. Although R3 may theoretically overlap with the ground truth (GT), the solid line style of R3 should not be completely obscured by the dashed line style of GT.}  
    An illustration of the Progressive Zoom-in Refinement mechanism on a horizontal
    bar chart and a donut chart. Successive regions (R1, R2, R3) show how the
    model iteratively crops closer to the target region, enabling increasingly
    precise numerical inference.
    }

  \label{fig:Refinement}
\end{figure}

\subsection{Iterative Visual Feedback}

With the axis priors already extracted, this component introduces an explicit form of 
visual feedback by overlaying the model's previous prediction onto the original 
chart. The key idea is to render the predicted data position back into the chart 
space so that the model can visually compare its estimate against the true 
geometric layout. This prediction-overlay mechanism provides a direct and 
interpretable hint that constrains subsequent inference and reduces ambiguity in 
the model's internal representation of the chart.

Given a prediction $p_{t}$ obtained from the previous iteration, we project 
$p_{t}$ back to chart coordinates using the extracted axis mapping and generate a 
visual overlay that marks the predicted location. This feedback overlay is defined as
\begin{equation}
F_{t} = \mathcal{O}(I, p_{t}, P_{\text{axis}}),
\end{equation}
where $\mathcal{O}$ denotes the overlay operator and $P_{\text{axis}}$ denotes the axis priors. The operator renders a chart-type–specific visual cue (e.g., marker, line, or angular indicator) at the predicted position. The resulting feedback-augmented visualization $F_t$ is then fed into the MLLM for the next iteration.

This visual feedback serves two complementary functions. First, it makes the 
model's own uncertainty explicit by exposing the discrepancy between the predicted 
and the actual data location in the chart. Second, it creates a stable and 
geometry-aligned visual anchor that guides the model to re-evaluate its earlier 
prediction and adjust it toward the correct value. As a result, the 
prediction-overlay feedback forms a self-correcting loop that enhances the model's 
ability to align numerical inference with the chart’s spatial encoding.



\subsection{Progressive Zoom-in Refinement}
To further elevate numerical inference accuracy, we introduce a 
\textit{Self-evolving Visual Refinement} mechanism that performs progressively 
localized zoom-in reasoning based on axis priors anchored in the preceding component. 
These axis-aware geometric priors provide a calibrated coordinate frame that 
supports a more disciplined and increasingly discriminative localization process.

Each refinement iteration treats the model's previous prediction as an endogenous 
cue, effectively a self-generated supervisory signal that guides the construction of 
tighter visual constraints through localized cropping, magnification, and grid 
densification. We formalize these operations through an iterative region-refinement 
operator. Let $p_{t}$ denote the model's numerical prediction at iteration $t$. The 
refined visual region is defined as

\begin{equation}
R_{t} = \mathcal{G}\big(\mathcal{M}(\mathcal{C}(I, p_{t}))\big),
\label{eq:refine-region-3.2.3}
\end{equation}

where $\mathcal{C}(I, p_{t})$ crops a region centered at the current predicted location $p_{t}$, $\mathcal{M}(\cdot)$ scales this crop, and $\mathcal{G}(\cdot)$ overlays a denser grid based on the axis priors.


Given the refined region $R_{t}$, the model updates its prediction through

\begin{equation}
p_{t+1} = f_{\theta}(R_{t}),
\label{eq:update-prediction-3.2.3}
\end{equation}

where $f_{\theta}$ denotes the multimodal model.

To ensure correctness, a self-verification criterion is introduced to check whether 
the cropped region still encloses the target chart element:

\begin{equation}
\mathcal{V}(R_{t}) =
\begin{cases}
1, & \text{if the target element lies within } R_{t}, \\
0, & \text{otherwise}.
\end{cases}
\label{eq:verification-3.2.3}
\end{equation}

If $\mathcal{V}(R_{t}) = 0$, the cropping region is adaptively expanded and 
re-centered before refinement continues. Formally, the system updates

\begin{equation}
R_{t} \leftarrow \mathcal{C}\big(I, p_{t}; \alpha \cdot s_{t}\big),
\label{eq:adaptive-expand-3.2.3}
\end{equation}

where $s_{t}$ is the current crop scale and $\alpha > 1$ controls the expansion 
ratio. Only when $\mathcal{V}(R_{t}) = 1$ does the algorithm proceed to compute 
$p_{t+1}$.

The complete refinement sequence forms a perceptual narrowing trajectory

\begin{equation}
p_{0} \rightarrow p_{1} \rightarrow \cdots \rightarrow p_{T},
\end{equation}

which empirically converges to a high-confidence estimate $p^{*}$ that minimizes 
prediction error:

\begin{equation}
p^{*} = \arg\min_{p} \|p - p_{\text{true}}\|.
\end{equation}
In practice, we run at most $T = 3$ refinement iterations.
% , or stop earlier if the predicted value changes by less than $\epsilon$.\pyl{No basis for determining the value of $\epsilon$ is defined.}
Through the interplay between self-evolving localization and verification-driven 
correction, this refinement component produces progressively tighter value estimates and 
steers the model toward precise and reliable numerical predictions.






% \begin{figure}[tb]
%   \centering
%   \includegraphics[width=\columnwidth, alt={XXXX.}]{figs/pipeline.png}
%   \caption{%
%   	The pipeline of our \toolName{}.
%   }
%   \label{fig:piplineFig}
% \end{figure}

\section{Experiments}
\label{sec:experiments}
\subsection{Experimental Setup}

% Existing chart datasets often suffer from overly simplified and homogeneous visual structures, resulting in insufficient richness and complexity. 
% To enhance the visual understanding capability of MLLMs, datasets such as \textit{ChartQA}\cite{ChartQA} introduce specific numerical annotations into charts. However, the explicit presence of numerical values may bias the model toward textual cues and hinder its ability to align visual structures with semantics. In contrast, \textit{PlotQA}\cite{PlotQA} does not add specific numerical annotations, but its coverage of chart types is too limited, which leads to a lack of overall diversity. Although some datasets, for example \textit{ChartQAPro}\cite{ChartQAPro}, \textit{ChartX}\cite{ChartX}, \textit{ChartLlama}\cite{Chartllama}, and \textit{EvoChart}\cite{EvoChart}, offer a more comprehensive coverage of chart types, they still lack the most critical visual localization information for our study—namely, tick marks.
\subsubsection{\textbf{Evaluation Datasets}}
% To assess the effectiveness of our proposed \toolName{}, we evaluate it on dataset that generated by ourself across various chart types, including bar, line, scatterplot, bubble, pie, donut, radar, and rose charts. We introduce multi-dimensional visual parameters for each chart type, such as bar orientation, line style, label position, sector spacing, and area fill patterns, to ensure the diversity that aligned with the real-world charts, specifically as shown in \cref{tab:chart_types}. We know that there exists several benchmarks used for chart data extraction or chartQA, icluding ChartQA, PlotQA, and ExcelChart400K collected in ChartOCR, however, chart imaes in ChartQA most expose the data value on the original image, PlotQA only includes the scatterplot that weak in chart diversity, in the same time, ExcelChart400K collected in ChartOCR only include bar, line, and pie charts. In addition, ecxcept for the evalution for data extraction, we also have to evaluate the effectiveness of axis priors extraction, we generate our own dateset, that extend the chart types to include cartesian and polar axis, including  bar, line, scatterplot, bubble, pie, donut, radar, and rose charts, that record the axis priors information and ground data values. To augment the diversity of our datasets to more align with the real-world charts, we further introduced variations in canvas size, resolution, color schemes, font sizes, and the number of entities, thereby simulating the visual complexity and diversity encountered in real-world scenarios as faithfully as possible.
To rigorously evaluate \toolName{}, we construct a unified evaluation dataset spanning two coordinate families: Cartesian charts, including vertical and horizontal bar, line, scatterplot, and bubble charts, and polar charts, including pie, donut, radar, and rose charts. Each chart is paired with pixel-accurate ground truth values and pixel-level representations of axis priors, enabling joint evaluation of numerical value extraction and axis-aware understanding. Existing benchmarks such as ChartQA\cite{ChartQA}, PlotQA\cite{PlotQA}, and ExcelChart400K\cite{ChartOCR} are mainly designed for chart question answering or chart OCR tasks. For our task setting, differences in three aspects make these benchmarks difficult to use: limited chart type coverage, the presence of explicit numeric labels within charts, and the lack of ground truth for both underlying ground truth of data values 
% and pixel-level axis priors 
needed to evaluate numerical inference across diverse chart families. 
% To rigorously evaluate \toolName{}, we construct a comprehensive dataset that spans two coordinate families: Cartesian charts (bar, line, scatter, and bubble) and polar charts (pie, donut, radar, and rose).
% To rigorously evaluate \toolName{}, we construct a comprehensive
% dataset spanning a wide range of visualization
% styles and axis systems. \wy{1. axis systems? 2. ``A wide range of'' may be an overclaim, and we can explicitly mention the exact number of visualization types instead. } 

% Existing benchmarks such as ChartQA\cite{ChartQA}, PlotQA\cite{PlotQA},
% and ExcelChart400K\cite{ChartOCR} are mainly designed for chart question answering
% %
% or chart data extraction. \wy{Do we want to emphasize this? Our task is also chart data extraction, so the logic flow here is a bit weird.}
% %
% They either expose explicit numeric labels in the chart
% (ChartQA), provide limited chart diversity such as only scatterplots
% (PlotQA), or cover only bar, line, and pie charts (ExcelChart400K). 
% None of these datasets provides axis-prior ground truth, nor do they
% cover a sufficiently diverse set of chart types, especially those in
% the polar coordinate system. These two limitations prevent them from
% supporting the axis-aware and multi-type evaluation required by our
% visual hint prompting strategy.
% \wy{One possible missing point here is the availability of the ground truth value of the underlying data, which is crucial for evaluating the performance of our approach.}
% We construct a unified evaluation dataset that
% spans five Cartesian chart types (vertical bar, horizontal bar, line, scatter, and bubble) and four polar chart types (pie, donut, radar, and rose). Each chart is paired with pixel-accurate ground truth data values and axis-prior annotations, enabling joint evaluation of value extraction and axis understanding. 
As shown in Table \ref{tab:chart_types}, the dataset contains
450 charts (50 per type), and each chart includes several visual
targets such as bars, points, or sectors. In total, this yields over 4,000 point-level prediction instances, providing sufficient statistical stability for per-type evaluation while keeping the computational cost of multi-round MLLM inference tractable.
To approximate the visual and structural diversity of real-world
graphics, we introduce controlled variations in canvas size, image
resolution, color palettes, font families and sizes, label layouts,
object counts, and chart-specific properties such as bar orientation,
line and marker styles, bubble label positions, sector spacing, ring
widths, and radar dimensions. All charts are programmatically rendered
within a unified pipeline; the “Real” and “Synthetic” labels in Table
\ref{tab:chart_types} refer only to the provenance of the underlying
numeric data rather than to visual rendering. Real charts are generated
from real-world numeric tables, whereas synthetic charts rely on
randomly sampled distributions with controlled statistical patterns.
This design ensures consistent visual rendering while preserving
semantic variability across data sources. Table \ref{tab:chart_types} summarizes the chart categories, real or synthetic composition, and key style variations incorporated in our dataset, which is available at \href{https://github.com/tu4nzii/VisHintPrompt_datasets}{https://github.com/tu4nzii/VisHintPrompt\_datasets}. 
% \wy{1. Do we plan to publish the dataset? It will be good if we can; 2. It should be ``Table I'', NOT ``table I''. Please revise it accordingly.}



\begin{table}[htbp]
\centering
\caption{Overview of Chart Types and Data Composition}
\label{tab:chart_types}
\resizebox{\linewidth}{!}{%
\begin{tabular}{@{}llrrrl@{}}
\toprule
\textbf{Category} & \textbf{Type} & \textbf{Total} & \textbf{Real} & \textbf{Syn.} & \textbf{Variations} \\
\midrule
\multirow{5}{*}{Cartesian} 
& Vertical Bar         & 50 & 25 & 25 & 1. Single/Multi bars\\
& Horizontal Bar         & 50 & 25 & 25 & 1. Single/Multi bars\\
& Line        & 50 & 25 & 25 & 1. Vertex shape; 2. Line style \\
& Scatterplot     & 50 & 25 & 25 & 1. Vertex shape; 2. Labels \\
& Bubble      & 50 & 25 & 25 & 1. Label position \\
\cmidrule{1-6}
\multirow{5}{*}{Polar} 
& Pie          & 50 & 25 & 25 & --- \\
& Rose         & 50 &  0 & 50 & 1. Sector spacing \\
& Radar        & 50 &  0 & 50 & 1. Dimensions; 2. Fill \\
& Donut        & 50 &  0 & 50 & 1. Ring width \\
\bottomrule
\end{tabular}%
}
\end{table}


% \subsubsection{Evaluation Metrics}
% To comprehensively evaluate the performance of the proposed visual hint-guided reasoning system \textbf{VisHintPrompt} on coordinate prediction tasks, we adopt two complementary metrics.

% \textbf{(1) Mean Absolute Error (MAE).} This metric quantifies the semantic-level accuracy by measuring the average absolute difference between the predicted values and the ground truth values (e.g., chart values aligned with axes). For each data point, MAE is computed only on the available axis dimensions (e.g., Y-axis for bar charts, both X and Y for scatter plots). It is defined as:
% \begin{equation}
% \mathrm{MAE} = \frac{1}{D} \sum_{d \in \mathcal{A}} \frac{1}{N} \sum_{i=1}^{N} \left| \widehat{v}_i^{(d)} - v_i^{(d)} \right|
% \end{equation}
% where $\mathcal{A}$ denotes the set of available axes (e.g., $\{x,y\}$ for scatter plots, $\{y\}$ for bar charts), $D = |\mathcal{A}|$ is the number of available dimensions, and $N$ is the number of points.

% \textbf{(2) Relative Pixel Error (RE).} This metric evaluates visual alignment by measuring the average relative distance between predicted and ground-truth pixel coordinates, normalized by the image size. Similar to MAE, the RE is computed only on axes present in each task:
% \begin{equation}
% \mathrm{RE} = \frac{1}{D} \sum_{d \in \mathcal{A}} \frac{1}{N} \sum_{i=1}^{N} \frac{|\hat{p}_i^{(d)} - p_i^{(d)}|}{S^{(d)}}
% \end{equation}
% where $\hat{p}_i^{(d)}$ and $p_i^{(d)}$ denote the predicted and ground-truth pixel coordinates on dimension $d$, and $S^{(d)}$ is the image width or height, depending on the axis.

% These two metrics jointly evaluate semantic accuracy and spatial alignment, while accounting for the dimensionality of each chart type.
\subsubsection{\textbf{Evaluation Metrics}}

% Since each chart type provides a fixed and semantically aligned set of data
% points, predicted values correspond directly to ground-truth values without
% requiring cross-point matching. Following established evaluation practices in
% chart understanding research \cite{PlotQA,ChartQA,UniChart}, we adopt
% a range-normalized numerical error formulation that enables consistent comparison
% across charts with different value scales. The evaluation proceeds hierarchically
% from point-level errors to chart-level aggregation and finally to category-level
% summaries.

% Since each chart type provides a fixed set of semantically aligned data points, predicted values admit a direct one-to-one correspondence with ground-truth values, without requiring cross-point matching. 
Following the prior chart understanding research~\cite{PlotQA,ChartQA,UniChart}, we adopt a metric called
% range-normalized numerical error metric 
\textit{range-normalized error}
to enable consistent comparison across various charts with different value scales. The evaluation is conducted hierarchically, starting from \textit{point-wise} errors to aggregated
% \textit{point-level} errors as chart-level and further type-level errors.
\textit{chart-level} errors as well as \textit{type-level} errors.


\textbf{Point-wise Range-normalized Error (RNE).}
For a chart $c$ with ground truth values $\{v_i\}$ and predictions
$\{\hat{v}_i\}$, the range-normalized point-wise error is
\begin{equation}
e^{\mathrm{RNE}}_{c,i} =
\frac{\left| \hat{v}_i - v_i \right|}
     {v_{\max,c} - v_{\min,c}}.
\end{equation}
This normalization maps all errors into the $[0,1]$ interval, enabling fair
cross-chart comparison and aligning with prior work that evaluates chart value
prediction in normalized numeric space \cite{PlotQA,tang2023vistext}.

\textbf{Point-wise Relative Error (RE).}
To complement range normalization with a scale-sensitive metric, we also report
the relative error with respect to the ground truth value:
\begin{equation}
e^{\mathrm{RE}}_{c,i} =
\frac{\left| \hat{v}_i - v_i \right|}
     {\max\!\left(\left| v_i \right|, \epsilon\right)},
\end{equation}
where $\epsilon$ is a small constant to avoid division by zero. This metric
captures the relative deviation from the true value and is particularly
informative when charts contain values with different magnitudes.

\textbf{Chart-level Error.}
For a given point-wise error metric $m \in \{\mathrm{RNE}, \mathrm{RE}\}$,
the chart-level error is obtained by averaging the point-level errors:
\begin{equation}
E^{(m)}_c = \frac{1}{K_c} \sum_{i=1}^{K_c} e^{(m)}_{c,i},
\end{equation}
where $K_c$ denotes the number of data points in chart $c$. This produces a
single error measure per chart that reflects overall numerical consistency
under metric $m$.

\textbf{Type-level Error.}
Since our evaluation is conducted independently for each chart type, the final
score for a chart type is computed, for each metric $m$, by averaging the
chart-level errors over all charts of that type (e.g., bar chart and scatterplot) in our testing dataset: 
% \wy{category or type? Make it consistent throughout the paper!}
\begin{equation}
E^{(m)}_{\mathrm{type}} =
\frac{1}{C_{\mathrm{type}}}
\sum_{c \in \mathrm{type}} E^{(m)}_c,
\end{equation}
% where $C_{\mathrm{type}}$ is the number of charts of that type. This
% type-specific aggregation follows the reporting protocols of recent unified
% chart understanding models \cite{UniChart, han2023chartllama} and avoids
% cross-type mixing.
where $C_{\mathrm{type}}$ denotes the number of charts of that type. We report type-specific aggregates following recent unified chart understanding models \cite{UniChart, han2023chartllama} to avoid cross-type mixing.

Overall, this point–chart–type hierarchy, combined with both range-normalized
(RNE) and relative-to-ground-truth (RE) errors, provides a scale-invariant,
type-consistent, and literature-aligned evaluation of numerical inference
performance for \toolName{}.







\subsubsection{\textbf{Model Selection}}
To evaluate the model-agnostic generality of \toolName{}, we apply our
prompting strategy to a diverse set of MLLMs spanning both
closed-source and open-source systems. This design allows us to examine whether
structured visual hints consistently enhance numeric inference ability across
heterogeneous architectures without any fine-tuning or model-specific
adaptation.

\paragraph{Closed-source MLLMs}
% \wy{closed-source? The more common description is proprietary?}
We evaluate \toolName{} across three closed-source MLLMs with advanced visual–language capabilities.
\begin{itemize}
    \item \textbf{GPT-4o} — a general-purpose vision–language model known for its
          robust perception and reasoning.
    \item \textbf{Gemini~2.0 Flash} — an efficient and high-throughput model with
          competitive visual understanding.
    \item \textbf{Gemini~2.5 Flash} — an upgraded version with improved
          fine-grained spatial reasoning.
\end{itemize}

\paragraph{Open-source MLLMs}
We further evaluate two representative open-source models covering a wide range
of parameter scales and architectural families:
\begin{itemize}
    \item \textbf{InternVL3-78B} — a large, high-capacity open-source VLM with
          strong visual recognition.
    \item \textbf{Pixtral-12B-2409} — a transformer-based multimodal model with
          competitive image understanding.
    % \item \textbf{GLM-4.6V-Flash} — a lightweight model suitable for
    %       evaluating low-resource settings.
    
    % \item \textbf{Qwen2.5-VL-32B-Instruct} — a stronger variant offering improved
    %       geometric and numerical perception.
\end{itemize}

\paragraph{Rationale}
Our focus is to examine how much numeric extraction accuracy can be gained
\emph{without} any training, finetuning, or synthetic chart supervision. Modern
MLLMs already possess latent spatial and geometric reasoning capabilities, and
\toolName{} provides structured chart-specific visual hints that enable these
capabilities to be more effectively activated and utilized.
% \paragraph{Exclusion of chart-specific and OCR methods}
% \pyl{Conflict with the OCR method mentioned earlier in III.A.1)‘Alignment of tick values to pixel positions’}}
Chart-specific QA models (e.g., PlotQA, ChartQA, UniChart) rely on synthetic
training and support only a narrow set of chart families, making them
incompatible with our full-range numeric inference setting. 
% OCR-based pipelines are also excluded, as they require explicit printed numerical labels and cannot operate on most charts in our benchmark.

Overall, this model selection covers a comprehensive spectrum of MLLMs and
provides a rigorous testbed for assessing the generality and effectiveness of
\toolName{}.






\subsubsection{\textbf{Implementation Details}}


For closed-source
models, we access GPT-4o, Gemini-2.0-Flash, and Gemini-2.5-Flash via their
official APIs; GPT-4o uses deterministic decoding (temperature~$=0$), while the
Gemini models follow their default inference settings. For the open-source model Pixtral-12B-2409, inference is performed locally on a BI-V150 GPU server using the official released implementation, without any parameter modification. In contrast, InternVL3-78B 
% and GLM-4.6V-Flash
% \pyl{Is GLM omitted here?} 
is accessed via their official APIs under default inference settings. All prompts follow fixed templates for each elicitation component, and each model prediction is parsed through a strict JSON schema.
% with one retry if a malformed response is encountered. 
% \pyl{Retrying without modifying the input while setting the temperature to 0 is illogical, as the output will remain fixed and retain the original error. Instead, a proprietary closed-source LLM with non-zero temperature (and low formatting error propensity) is deployed for output format calibration.}
We did not perform any prompt tuning or model-specific adaptation beyond these fixed templates. The full implementation code is publicly available at \href{https://github.com/tu4nzii/VisHintPrompt_code}{https://github.com/tu4nzii/VisHintPrompt\_code}.








\begin{table*}[htbp]
\centering
\caption{
Main experimental results across five MLLMs on nine chart types.
We report Range-Normalized Error (RNE) and Relative Error (RE; lower is better) for both 
the baseline (original charts) and our VisHintPrompt method.
}
\label{tab:mllms_comparison}
\small
\setlength{\tabcolsep}{3.6pt}
\renewcommand{\arraystretch}{1.22}

\begin{adjustbox}{max width=\textwidth}
\begin{tabular}{
    m{2.8cm}<{\centering} 
    m{1.9cm}<{\centering} 
    % *{20}{S[table-format=2.2]}
    *{18}{S[table-format=2.2]}
    | % ← 只在这里加
    *{2}{S[table-format=2.2]}
}
\toprule
\multirow{2}{*}{\textbf{MLLMs}} &
\multirow{2}{*}{\textbf{Setting}} &
\multicolumn{2}{c}{Scatterplot} &
\multicolumn{2}{c}{Bubble} &
% \multicolumn{2}{c}{v$-$Bar} &
% \multicolumn{2}{c}{h$-$Bar} &
\multicolumn{2}{c}{Vertical Bar} &
\multicolumn{2}{c}{Horizontal Bar} &
\multicolumn{2}{c}{Line} &
\multicolumn{2}{c}{Radar} &
\multicolumn{2}{c}{Rose} &
\multicolumn{2}{c}{Pie} &
\multicolumn{2}{c}{Donut} &
\multicolumn{2}{|c}{Avg.} \\



\cmidrule(lr){3-4}   % Scatter
\cmidrule(lr){5-6}   % Bubble
\cmidrule(lr){7-8}   % v-Bar
\cmidrule(lr){9-10}  % h-Bar
\cmidrule(lr){11-12} % Line
\cmidrule(lr){13-14} % Radar
\cmidrule(lr){15-16} % Rose
\cmidrule(lr){17-18} % Pie
\cmidrule(lr){19-20} % Donut
\cmidrule(lr){21-22} % Avg
& & {RNE} & {RE} & {RNE} & {RE} & {RNE} & {RE} & {RNE} & {RE} &
      {RNE} & {RE} & {RNE} & {RE} & {RNE} & {RE} & {RNE} & {RE} & {RNE} & {RE}  & {RNE} & {RE}\\
\cmidrule(lr){1-2}   % MLLMs
\cmidrule(lr){3-4}   % Scatter
\cmidrule(lr){5-6}   % Bubble
\cmidrule(lr){7-8}   % v-Bar
\cmidrule(lr){9-10}  % h-Bar
\cmidrule(lr){11-12} % Line
\cmidrule(lr){13-14} % Radar
\cmidrule(lr){15-16} % Rose
\cmidrule(lr){17-18} % Pie
\cmidrule(lr){19-20} % Donut
\cmidrule(lr){21-22} % Avg

% \midrule

% GPT-4o
\multirow{2}{*}{GPT-4o}
& Baseline 
    & 6.72 & 4.69 & 6.63 & 25.34 & 5.7 & 14.22 & 9.05 & 49.44 & 3.84 & 9.77 & 25.54 & 55.71 & 16.58 & 22.64 & \textbf{3.72} & \textbf{25.72} & \textbf{2.64} & \textbf{30.93} & 8.94 & 26.5\\
& \cellcolor{gray!15}VisHintPrompt
    & \cellcolor{gray!15}\textbf{3.44} & \cellcolor{gray!15}\textbf{2.23} 
    & \cellcolor{gray!15}\textbf{3.38} & \cellcolor{gray!15}\textbf{10.90}
    & \cellcolor{gray!15}\textbf{3.03} & \cellcolor{gray!15}\textbf{6.91}
    & \cellcolor{gray!15}\textbf{1.94} & \cellcolor{gray!15}\textbf{9.22}   %h-bar
    & \cellcolor{gray!15}\textbf{2.13} & \cellcolor{gray!15}\textbf{4.11} %line
    & \cellcolor{gray!15}\textbf{11.92} & \cellcolor{gray!15}\textbf{32.14}
    & \cellcolor{gray!15}\textbf{9.99} & \cellcolor{gray!15}\textbf{13.4}
    & \cellcolor{gray!15}12.52 & \cellcolor{gray!15}43.81
    & \cellcolor{gray!15}7.94 & \cellcolor{gray!15}67.15
    & \cellcolor{gray!15}\textbf{6.25} & \cellcolor{gray!15}\textbf{21.1}\\

% Gemini 2.0 Flash
\multirow{2}{*}{Gemini 2.0 Flash}
& Baseline 
     & 3.08 & 13.43 & 3.03 & 12.79 & \textbf{0.67} & \textbf{0.26} & 5.43 & 14.03 & \textbf{0.31} & \textbf{1.02} & 20.63 & 50.21 & 18.9 & 25.94 & 3.8 & 23.53 & 2.81 & 27.84 & 6.52 & 18.78\\
& \cellcolor{gray!15}VisHintPrompt
    & \cellcolor{gray!15}\textbf{1.38} & \cellcolor{gray!15}\textbf{4.30}
    & \cellcolor{gray!15}\textbf{1.41} & \cellcolor{gray!15}\textbf{4.06}
    & \cellcolor{gray!15}1.65 & \cellcolor{gray!15}0.64
    & \cellcolor{gray!15}\textbf{1.65} & \cellcolor{gray!15}\textbf{4.56} %h-bar
    & \cellcolor{gray!15}0.81 & \cellcolor{gray!15}1.66 %line
    & \cellcolor{gray!15}\textbf{6.93} & \cellcolor{gray!15}\textbf{21.77}
    & \cellcolor{gray!15}\textbf{10.0} & \cellcolor{gray!15}\textbf{13.57}
    & \cellcolor{gray!15}\textbf{3.03} & \cellcolor{gray!15}\textbf{15.63}
    & \cellcolor{gray!15}\textbf{1.75} & \cellcolor{gray!15}\textbf{16.92}
    & \cellcolor{gray!15}\textbf{3.18} & \cellcolor{gray!15}\textbf{9.23}\\

% Gemini 2.5 Flash
\multirow{2}{*}{Gemini 2.5 Flash}
& Baseline 
    & 3.19 & 9.44 & 3.31 & 10.37 & 2.21 & 5.49 & 3.41 & 37.02 & \textbf{0.54} & \textbf{1.84} & 27.46 & 52.0 & 12.23 & 16.83 & 2.92 & 13.83 & 2.45 & 21.56 & 6.41 & 19.79 \\
& \cellcolor{gray!15}VisHintPrompt
    & \cellcolor{gray!15}\textbf{2.62} & \cellcolor{gray!15}\textbf{7.20}
    & \cellcolor{gray!15}\textbf{3.10} & \cellcolor{gray!15}\textbf{7.89}
    & \cellcolor{gray!15}\textbf{0.95} & \cellcolor{gray!15}\textbf{2.78}
    & \cellcolor{gray!15}\textbf{0.77} & \cellcolor{gray!15}\textbf{5.42} %h-bar
    & \cellcolor{gray!15}1.84 & \cellcolor{gray!15}3.56 %line chart
    & \cellcolor{gray!15}\textbf{6.63} & \cellcolor{gray!15}\textbf{21.6}
    & \cellcolor{gray!15}\textbf{7.16} & \cellcolor{gray!15}\textbf{10.45}
    & \cellcolor{gray!15}\textbf{1.89} & \cellcolor{gray!15}\textbf{9.07}
    & \cellcolor{gray!15}\textbf{1.47} & \cellcolor{gray!15}\textbf{13.38}
    & \cellcolor{gray!15}\textbf{2.94} & \cellcolor{gray!15}\textbf{9.04}\\

% InternVL3-78B
\multirow{2}{*}{InternVL3-78B}
& Baseline 
    & 3.30 & 2.19 & 5.04 & 12.47 & \textbf{1.27} & 4.35 & 5.39 & 16.69 & \textbf{0.83} & \textbf{2.07} & 16.25 & 37.0 & 11.73 & 16.97 & \textbf{3.62} & 25.72 & \textbf{2.28} & 29.89 & 5.52 & 16.37\\
& \cellcolor{gray!15}VisHintPrompt   % Acc / RE
    & \cellcolor{gray!15}\textbf{1.45} & \cellcolor{gray!15}\textbf{1.01}
    & \cellcolor{gray!15}\textbf{3.79} & \cellcolor{gray!15}\textbf{9.95} % bubble
    & \cellcolor{gray!15}1.40 & \cellcolor{gray!15}\textbf{4.19}  % V-Bar
    & \cellcolor{gray!15}\textbf{0.64} & \cellcolor{gray!15}\textbf{5.43}    % H-bar
    & \cellcolor{gray!15}1.77 & \cellcolor{gray!15}3.79    % Line
    & \cellcolor{gray!15}\textbf{6.78} & \cellcolor{gray!15}\textbf{15.85} % radar
    & \cellcolor{gray!15}\textbf{8.97} & \cellcolor{gray!15}\textbf{13.09} % rose
    & \cellcolor{gray!15}4.94 & \cellcolor{gray!15}\textbf{18.73}    % pie
    & \cellcolor{gray!15}3.02 & \cellcolor{gray!15}\textbf{26.95}
    & \cellcolor{gray!15}\textbf{3.64} & \cellcolor{gray!15}\textbf{11.0}\\    % donut

% Pixtral-12B-2409
\multirow{2}{*}{Pixtral-12B-2409}
& Baseline 
    & 7.38 & 4.86 & 6.69 & 18.92 & \textbf{2.03} & \textbf{3.27} & \textbf{6.98} & 43.43 & \textbf{2.62} & 7.38 & 33.15 & 72.36 & 34.57 & 46.68 & \textbf{6.13} & 55.04 & 9.73 & 84.60 & 12.14 & 37.4 \\
& \cellcolor{gray!15}VisHintPrompt   % Acc / RE
    & \cellcolor{gray!15}\textbf{4.73} & \cellcolor{gray!15}\textbf{3.22}
    & \cellcolor{gray!15}\textbf{6.36} & \cellcolor{gray!15}\textbf{9.32} % bubble
    & \cellcolor{gray!15}15.26 & \cellcolor{gray!15}24.6  % V-Bar
    & \cellcolor{gray!15}9.28 & \cellcolor{gray!15}\textbf{36.29}    % H-bar
    & \cellcolor{gray!15}2.95 & \cellcolor{gray!15}\textbf{6.99}    % Line
    & \cellcolor{gray!15}\textbf{15.56} & \cellcolor{gray!15}\textbf{44.2}      % radar
    & \cellcolor{gray!15}\textbf{11.89} & \cellcolor{gray!15}\textbf{17.16} % rose
    & \cellcolor{gray!15}13.31 & \cellcolor{gray!15}\textbf{50.28}    % pie
    & \cellcolor{gray!15}\textbf{6.15} & \cellcolor{gray!15}\textbf{44.82} 
    & \cellcolor{gray!15}\textbf{9.50} & \cellcolor{gray!15}\textbf{26.32}\\    % donut

\midrule
\multirow{2}{*}{\shortstack{Avg.}}
& Baseline 
    & 4.73 & 6.92 & 4.94 & 15.98 & \textbf{2.38} & \textbf{5.52} & 6.05 & 32.12 & \textbf{1.63} & 4.42 & 24.61 & 53.46 & 18.8 & 25.81 & \textbf{4.04} & 30.71 & \textbf{3.98} & 38.96 & 7.91 & 23.77 \\
& \cellcolor{gray!15}VisHintPrompt
    & \cellcolor{gray!15}\textbf{2.72} & \cellcolor{gray!15}\textbf{3.59}
    & \cellcolor{gray!15}\textbf{3.61} & \cellcolor{gray!15}\textbf{8.42}
    & \cellcolor{gray!15}4.46 & \cellcolor{gray!15}7.82
    & \cellcolor{gray!15}\textbf{2.86} & \cellcolor{gray!15}\textbf{12.18}
    & \cellcolor{gray!15}1.9 & \cellcolor{gray!15}\textbf{4.02}
    & \cellcolor{gray!15}\textbf{9.56} & \cellcolor{gray!15}\textbf{27.11}
    & \cellcolor{gray!15}\textbf{9.6} & \cellcolor{gray!15}\textbf{13.53}
    & \cellcolor{gray!15}7.14 & \cellcolor{gray!15}\textbf{27.5}
    & \cellcolor{gray!15}4.07 & \cellcolor{gray!15}\textbf{33.84} 
    & \cellcolor{gray!15}\textbf{5.1} & \cellcolor{gray!15}\textbf{15.34}\\

\bottomrule
\end{tabular}
\end{adjustbox}
\end{table*}

\section{Results and Analysis}

\subsection{Quantitative Evaluation}

This section presents the quantitative performance of MLLMs under our
proposed \emph{\toolName{}}. The evaluation covers more than 4,000
point-level prediction instances across Cartesian and polar chart families.
Unless otherwise specified, all reported errors are RNE and RE as defined in Section~\ref{sec:experiments}, obtained by averaging point-wise errors over charts within each
chart type and reported as percentages.  
Importantly, none of the benchmark charts include explicit numeric labels
on the visual marks, precluding trivial solutions based on OCR.
% \pyl{Conflict with the OCR method mentioned earlier in III.A.1)‘Alignment of tick values to pixel positions’}
Consequently, all predictions must rely on visual inference from geometric
cues such as bar height, scatter-point position, angular extent, and polar
distance.

\subsubsection{\textbf{Overall Performance}}
As shown in Table~\ref{tab:mllms_comparison}, averaged over five MLLMs and nine chart types, \toolName{} leads to overall reductions in numerical error (RNE and RE) compared to the baseline. These reductions are observed across all five evaluated MLLMs.
% As shown in Table~\ref{tab:mllms_comparison}, \toolName{} consistently improves numerical accuracy across all evaluated MLLMs.
% \wy{1. How do you know this from Table II? 2.VisHintPrompt is NOT always for all (see pie and donut chart).}
%
%
Specifically, across all model–chart type combinations, type-level RNE and RE decrease in 32 of the 45 cases, with closed-source MLLMs achieve improvements on at least 7 of the 9 chart types, while open-source MLLMs also improve on at least 5 of the 9. 
% Across model-chart combinations, type-level RNE decreases in 38 of 54 cases, with no systematic degradation observed in the remainder. \wy{what do you mean by ``no systematic degradation''? error or accuracy?}
% These gains are consistent across both RNE and RE and are of practical magnitude.


% As shown in Table~\ref{tab:mllms_comparison}, \toolName{} consistently
% improves numerical accuracy across all evaluated MLLMs. Considering all
% model--chart combinations, type-level RNE decreases under \toolName{} on
% 38 out of 54 combinations, and we do not observe any systematic degradation
% on the remaining cases. These improvements are reflected in both RNE and RE
% and are typically of practical magnitude.

Closed-source MLLMs achieve the lowest post-prompting type-level RNE and
RE. For example, for Gemini-2.5-Flash, the mean type-level RNE across chart
types drops from 6.41\% to 2.94\% and the mean type-level RE from 19.79\%
to 9.04\%, reflecting its stronger visual--language integration.
Open-source models, in contrast, often exhibit larger \emph{relative}
reductions in error. For a representative open-source model,
Pixtral-12B-2409, the mean type-level RNE across chart
types drops from 12.14\% to 9.5\% and the mean type-level RE from 37.4\%
to 26.32\%, suggesting that scaffolded visual hints can
partially compensate for weaker geometric priors rather than merely
amplifying existing strengths. The fact that these gains appear across
heterogeneous architectures indicates that the proposed strategy behaves
largely model-agnostically.
% For a representative open-source model
% (Pixtral-12B-2409), the type-level RNE on structurally demanding chart types
% such as horizontal bar, bubble, and donut charts is reduced by roughly
% $16.44\%\text{--}47.02\%$, suggesting that scaffolded visual hints can
% partially compensate for weaker geometric priors rather than merely
% amplifying existing strengths. The fact that these gains appear across
% heterogeneous architectures indicates that the proposed strategy behaves
% largely model-agnostically.

\textbf{Overall, \toolName{} reduces numerical error across five MLLMs and nine chart types, with lower chart-averaged errors observed for each open-source and closed-source MLLM.}
Taken together, these results confirm the effectiveness of \toolName{} in improving numerical inference from charts across diverse chart types and MLLMs.

% Overall, the aggregate results demonstrate that \toolName{} delivers robust
% and consistent accuracy improvements across 5 open- and closed- source MLLMs, providing a strong
% basis for examining how these gains vary across specific chart families.
% \wy{Highlight the key conclusion or sentence(s) in this paragraph.}

% \subsection{Quantitative Evaluation}

% This section presents the quantitative performance of MLLMs under our proposed \emph{\toolName{}}. The evaluation covers more than 4,000 point-level prediction instances across Cartesian and polar chart families. Unless otherwise specified, all reported errors are type-level
% Range-Normalized Error (RNE) and Relative Error (RE) as defined in
% Section~4, obtained by averaging point-wise errors over charts within each
% chart type and reported as percentages.  
% Importantly, none of the benchmark charts include explicit numeric labels on the visual marks, precluding trivial solutions based on OCR. Consequently, all predictions must rely on visual inference from geometric cues such as bar height, scatter-point position, angular extent, and polar distance.

% \subsubsection{Overall Performance}

% As shown in Table~\ref{tab:mllms_comparison}, \toolName{} consistently
% improves numerical accuracy across all evaluated MLLMs. Considering all
% model--type combinations, type-level RNE decreases under \toolName{} on
% 50 out of 54 combinations, and we do not observe any systematic
% degradation on the remaining cases. These improvements are reflected in
% both RNE and RE and are typically of practical magnitude.

% Closed-source models achieve the lowest post-prompting type-level RNE and RE
% (e.g., for Gemini-2.5-Flash, the mean type-level RNE across chart types
% drops from 6.07\% to 2.95\% and the mean type-level RE from 15.78\% to
% 8.58\%), reflecting its stronger visual--language integration. Open-source
% models, in contrast, often exhibit larger \emph{relative} reductions in
% error. For a representative open-source model (Pixtral-12B-2409),
% type-level RNE on structurally demanding chart types such as horizontal
% bar, bubble, and donut charts is reduced by roughly
% $16.44\%\text{--}47.02\%$, suggesting that scaffolded visual hints can
% partially compensate for weaker geometric priors rather than merely
% amplifying existing strengths. The fact that these gains appear across
% heterogeneous architectures indicates that the proposed strategy behaves
% largely model-agnostically.

% Overall, the aggregate results demonstrate that \toolName{} delivers robust
% and consistent accuracy improvements at the model level, providing a strong
% basis for examining how these gains vary across specific chart families.


\subsubsection{\textbf{Performance Across Chart Types}}
% To analyze how \toolName{} performs across different chart types, 
We group the nine chart types into three categories for focused analysis: charts with high baseline accuracy (vertical bar and line charts), charts with visually complex structures (horizontal bar, scatterplot, bubble, radar, and rose charts), and charts involving proportion-based encodings, for which \toolName{} introduces a grid-based angle estimation strategy for proportion inference (pie and donut charts).


% To examine how visual design affects \toolName{}, we analyze performance across chart types, revealing a systematic difficulty structure in multimodal chart reasoning tied to visual regularity, spatial encoding, and geometric complexity.

% To understand how visual design shapes the effectiveness of \toolName{}, we
% next analyze performance across chart types. This analysis reveals a
% systematic difficulty structure in multimodal chart reasoning that is
% closely tied to the visual regularity, spatial encoding, and geometric
% demands inherent to each chart family.

\paragraph{Charts with high baseline accuracy}
\textbf{Highly prevalent vertical bar and line charts with dense, grid-aligned structures exhibit the strongest baseline performance across both closed- and open-source MLLMs, resulting in limited room for improvement under \toolName{}.}
 Most cases that do not show clear improvements arise from this group. For example, in one of the strongest closed-source models, Gemini-2.0-Flash, baseline errors on these chart types are already extremely low, with RNE below 0.67\% and 0.31\% and RE below 0.26\% and 1.02\% for vertical bar and line charts, respectively. Under such near-saturated conditions, \toolName{} yields only marginal changes, and minor fluctuations may occasionally appear. Model-specific differences can nevertheless be observed. While Pixtral-12B-2409 shows limited gains on vertical bar charts, GPT-4o and Gemini-2.5 Flash still benefit from \toolName{} due to their relatively higher baseline errors. In one specific case, Pixtral-12B-2409 on vertical bar charts, \toolName{} may introduce slight degradations, which can be explained by near-saturated baseline performance and sensitivity to additional visual cues.

% Vertical bar charts and line charts tend to achieve the strongest baseline
% performance across closed- and open- source MLLMs. 7 exception cases which don't achieve improvements are from this.
% For one of the strongest closed-source models
% (Gemini-2.0-Flash), baseline type-level errors on these two chart types are
% already extremely low, with RNE below 0.67\% and 0.31\% and RE below 0.26\%
% and 1.02\% on vertical bar and line charts, respectively. 
% Under such
% near-saturated conditions, \toolName{} produces negligible changes for
% MLLMs, including a few minor fluctuations that occasionally appear as
% slight degradations. But, the performance of Pixtral-12B-2409 on vetical bar charts is worse a lot. 
% In contrast, GPT-4o exhibits substantially higher
% baseline errors on both vertical bar and line charts, and Gemini 2.5 Flash on vertical bar, still benefit from our
% strategy. We explain this phenomenon by for the simple charts, the MLLMs most activate the chart OCR ability to conduct numerical inference, because already exist model able to achieve excellent estimation for vertical bar charts and line charts. The influence introducing by \toolName{} bring some negative.
% % : for these models, \toolName{} typically yields modest but consistent reductions in type-level RNE (on the order of 20\%--50\%). 
% Overall, the limited gains and occasional small regressions on
% bar and line charts are better explained by saturation effects than by a
% lack of usefulness of the visual hints.
% \wy{What are the core observations or conclusions here? It is a bit hard to follow. 
% Suggestions:
% 1. Highlight the key message/sentences;
% 2. Which part of Table II supports the observations?}
 



\paragraph{Structurally demanding chart families}
% \pyl{Why not directly adopt the term charts with visually complex structures from the preceding section 2) Performance Across Chart Types for consistency between the front and back content?}
\textbf{Structurally complex chart types, including horizontal bar, scatterplot, bubble, radar, and rose charts, exhibit substantially higher baseline errors, yet consistently benefit from \toolName{}, with pronounced error reductions observed across all evaluated open-source and closed-source MLLMs.}
When averaged across MLLMs, all five structurally demanding chart types exhibit approximately 50\% relative reductions in both RNE and RE compared to their baseline errors.
% Specifically, from the average results across MLLM types, we can see that these five type charts all achieve 50\% improvements relate to baseline estimation errors. 
At the chart–MLLM combination level, \toolName{} typically reduces type-level RNE by approximately 2\%–6\% and RE by 5\%–10\% for most models.
More pronounced improvements are observed for polar charts. On radar charts, type-level RNE across models decreases from approximately 25\%–70\% to 6\%–27\%, with a median reduction of around 8\%. Similarly, on rose charts, RNE is reduced from roughly 10\%–30\% to 7\%–10\%, again with median reductions close to 8\%. Comparable trends are observed for scatterplot and bubble charts.
Horizontal bar charts exhibit the largest gains. For example, two closed-source MLLMs (Gemini~2.0~Flash and Gemini~2.5~Flash) and one open-source MLLM (InternVL3-78B) reduce RE from approximately 10\%–40\% at baseline to around 5\% under \toolName{}.
Taken together, these results provide strong empirical evidence that \toolName{} effectively enhances numerical inference under challenging visual structures where baseline model performance is limited.
% At the chart-MLLM combination level, 
% % Across these structurally demanding families, 
% \toolName{} typically
% reduces type-level RNE and RE by roughly 2\%--6\%, and 5\%--10\% for most models. For
% example, on radar charts, type-level RNE across models decreases from
% approximately 25\%--70\% to 6\%--27\% (with a median reduction around
% 8\%), and on rose charts from 10\%--30\% to 7\%--10\% (median reduction
% around 8\%), with similar patterns observed for scatter and bubble charts. In particular, \toolName{} on horizontal bar charts achieves most improvements, in which two closed-source MLLMs, Gemini 2.0 Flash and Gemini 2.5 Flash, and a open-source MLLM, InternVL3-78B, all reduction RE from 10\%--40\% to around 5\% errors.
% % These patterns highlight that scaffolded visual hints are
% % particularly effective for chart families requiring non-trivial geometric
% % interpretation, such as angular reasoning, polar geometry, or multi-axis
% % alignment.
% These results provide strong empirical evidence that \toolName{} effectively enhances numerical inference under challenging visual structures where baseline model performance is limited.

\begin{figure*}[tb]
  \centering
  \includegraphics[width=\textwidth,
                   alt={case-2}]{figs/case1-1.png}
  \caption{Qualitative examples illustrating grid-aware enhancement and iterative visual feedback (Case~1) on three chart types: bubble, horizontal bar, and donut. Colored crosshairs denote predictions from different stages: baseline (blue), grid-aware enhancement (red, also the round-0 of visual feedback), and the first and second feedback rounds (purple and yellow). Grid enhancement reduces coarse localization ambiguity, while iterative feedback further corrects large residual deviations and stabilizes predictions across chart types. The corresponding numerical values for these examples are reported in Table~\ref{tab:case_results_feed}.}
  % \pyl{The category order here (b,a,c) conflicts with the predefined sequence (a,b,c) in Section V.A.2 and Fig.5. Please reorder to a,b,c for consistency.}
  % \pyl{Formula (6) in Section III and Fig. 3 both specify three rounds of feedback, but only two rounds are mentioned here. Is additional clarification needed?}

  \label{fig:case1}
\end{figure*}



\paragraph{Proportion-based charts with angle estimation}
\textbf{For proportion-based charts (pie and donut), \toolName{} introduces an angle-based proportion inference strategy under which Gemini-series MLLMs achieve substantial error reductions, revealing clear evidence of structured visual reasoning in this previously underexplored setting, while other models show less consistent or incomplete improvements.}
Quantitatively, Gemini-series MLLMs show consistent and substantial improvements on proportion-based charts under \toolName{}. For example, on pie and donut charts, Gemini~2.0~Flash reduces RE from 23.53\% to 15.63\% and RNE from 3.80\% to 3.03\%, while Gemini~2.5~Flash further reduces RE from 21.56\% to 13.38\% and RNE from 2.92\% to 1.89\%. 
% These consistent reductions across both metrics and model versions indicate effective adaptation to angle-based proportion inference enabled by \toolName{}.
In contrast, GPT-4o shows less consistent improvements on pie and donut charts under \toolName{}. A closer inspection reveals that its errors often stem from systematic misinterpretation of angular direction, particularly confusing clockwise and counterclockwise readings after sector-level zoom-in. This suggests that, for GPT-4o, orientation-sensitive angle reasoning remains a challenge. 
% For the two evaluated open-source MLLMs, \toolName{} consistently reduces RE on pie and donut charts, while improvements in RNE are less consistent. 
For the two evaluated open-source MLLMs, \toolName{} consistently reduces RE on pie and donut charts, indicating improved accuracy in estimating small proportions. In contrast, improvements in RNE are less consistent, suggesting that open-source models still face challenges in fully adapting to this new angle-based reasoning task. 
This improvement in RE is likely supported by the zoom-in refinement employed by \toolName{}, which appears to benefit the recognition of small targets, consistent with prior work showing that magnification can facilitate visual understanding. Taken together, these results highlight the effectiveness of \toolName{} for angle-based proportion inference, while also revealing meaningful differences in how current MLLMs adapt to this new visual reasoning setting.



% \textbf{Donut charts} depend on accurate estimation of angular boundaries, where small localization errors directly translate into value errors.

% \subsubsection{\textbf{Summary}}
% Taken together, the type-level results show that \toolName{} delivers its
% greatest benefits on chart families that impose complex geometric
% reasoning, while still offering modest but reliable refinements on visually
% simpler chart types. The resulting performance distribution delineates the
% visual structures that current multimodal models naturally handle well and
% those for which targeted visual prompting remains essential. 




% \subsubsection{Overall Performance}
% As shown in Table \ref{tab:mllms_comparison},
% across most models and chart families, the final visual-hint prompting strategy yields consistent and meaningful accuracy improvements. The magnitude of these gains varies with chart complexity.  
% Visually simpler charts---including vertical bar, line, and pie charts---already exhibit strong baseline performance due to their regular layouts and salient geometric cues. While the prompting strategy still delivers measurable refinement, the absolute improvement margins remain smaller because baseline accuracy is already high.

% In contrast, chart families with more complex geometric structures show substantially lower baseline accuracy and correspondingly larger error reductions. Closed-source models achieve the strongest absolute accuracy under the final strategy, reflecting their more advanced visual--language integration. Open-source models exhibit the largest proportional improvements, suggesting that structured visual hints help compensate for weaker geometric priors rather than merely reinforcing existing strengths.  
% The consistency of improvements across heterogeneous architectures indicates that the method is largely model-agnostic.

% Taken together, these results outline a stable performance landscape: the visual-hint prompting strategy reliably enhances predictions on chart types aligned with current model priors and provides substantial benefits on charts that impose more demanding geometric reasoning.

% \subsubsection{Performance Across Chart Types}

% A breakdown by chart type reveals a clear difficulty structure in multimodal chart reasoning. This structure emerges from the visual regularity, spatial encoding, and geometric demands inherent to each chart family.

% \paragraph{Charts with High Baseline Accuracy.}
% Pie charts, line charts, and vertical bar charts tend to achieve strong baseline accuracy. Their geometric regularity---clear axes, clean categorical boundaries, and straightforward value mappings---facilitates robust visual inference even in the absence of explicit numeric labels.  
% Although the visual-hint prompting strategy still yields measurable refinements, improvement margins remain limited because models already operate near saturation on these chart types.

% \paragraph{Structurally Demanding Chart Families.}
% More complex chart types exhibit markedly lower baseline accuracy and substantially larger gains under the proposed strategy:
% \begin{itemize}
%     \item \textbf{Horizontal bar charts} require reinterpreting directional mappings when the value axis is rotated.
%     \item \textbf{Donut charts} rely on accurate estimation of angular boundaries and color transitions.
%     \item \textbf{Rose charts} combine angular segmentation with polarly varying segment lengths, introducing irregular polar geometries that are difficult to parse.
%     \item \textbf{Radar charts} involve multiple polar axes with heterogeneous scales, requiring models to identify both the correct axis and the corresponding polar position.
% \end{itemize}

% Across these structurally complex families, the final visual-hint prompting strategy typically reduces error-over-range by approximately 30--50\% for both closed-source and open-source models. The magnitude and consistency of these gains highlight that structured visual induction is especially effective for chart types requiring non-trivial geometric interpretation.

% \paragraph{Summary.}
% These findings indicate that the proposed visual-hint prompting strategy delivers the greatest benefit on chart families that impose angular reasoning, polar geometry, or multi-axis alignment, while still offering modest refinement on visually simpler chart types. The resulting performance distribution highlights the types of visual structures multimodal models naturally manage well and those for which targeted prompting remains essential.












\subsection{Qualitative Analysis}
To illustrate how \toolName{} operates and how its components contribute to numerical inference, we present a qualitative analysis with two representative cases. Case~1 examines the effect of axis-aware grid enhancement and iterative visual feedback, showing how structured visual scaffolds progressively reduce localization ambiguity and stabilize value estimation. Case~2 focuses on the zoom-in refinement mechanism, illustrating how progressive amplification enables increasingly fine-grained value discrimination.




\subsubsection{\textbf{Case1. Grid-aware Enhancement and Iterative Visual Feedback}}
We qualitatively examine the effects of grid-aware enhancement and iterative visual feedback using three representative chart instances: a bubble chart, a bar chart, and a donut chart.
% We examine the effects of grid augmentation and iterative visual feedback using three representative chart types: bubble charts, bar charts, and donut charts. 
% As shown in Fig.~\ref{fig:case1}, predictions are encoded as colored crosshairs by inference stage: baseline (blue), grid-aware enhancement (red), and the first and second rounds of visual feedback (purple and yellow).
As shown in Fig.~\ref{fig:case1}, predictions are encoded as colored crosshairs by inference stage: baseline (blue), grid-aware enhancement as the initial (round-0) visual feedback (red), followed by the first and second rounds of visual feedback (purple and yellow).
% \pyl{Formula (6) in Section III and Fig. 3 both specify three rounds of feedback, but only two rounds are mentioned here. Is additional clarification needed?}
Overall, grid-aware enhancement improves prediction quality over the baseline across both bubble and horizontal bar charts. 
In the bubble chart case, the grid-enhanced predictions are consistently closer to the true bubble centers than the baseline predictions, indicating that all examined data points benefit from grid augmentation.
The horizontal bar chart contains multiple category groups. To avoid visual clutter and occlusion, we display only one representative group (Group~2) in the figure for clarity. 
% Grid-aware enhancement improves predictions for more than half of the data points. 
The grid-enhanced predictions align more closely with the true bar endpoints than the baseline, particularly for bars that exhibit larger estimation errors under the baseline setting. For example, bars corresponding to labels C and I (Fig.~\ref{fig:case1} (Horizontal Bar)) show noticeable over-estimation in the baseline predictions, while grid augmentation stabilizes these predictions within a much narrower error range.
% As shown in the Fig.~\ref{fig:case-feedback}, predictions from baseline, grid-aware enhancement and feedback rounds are encoded using colored crosshairs.
% % , following the same visual convention as in the actual iterative feedback process. 
% Specifically, predictions are color-coded by inference stage: baseline (blue), grid-aware enhancement (red), and the first and second rounds of visual feedback (purple and yellow).
% Specifically, blue denotes the baseline prediction, red corresponds to predictions made on grid-aware enhancement images, and purple and yellow indicate predictions obtained after the first and second rounds of visual feedback, respectively.


% For the horizontal bar chart, grid-aware enhancement improves predictions for more than half of the data points. Visually, the grid-enhanced predictions align more closely with the true bar endpoints than the baseline, particularly for bars located between adjacent tick marks or near the middle of the axis range, where length estimation is less well anchored under the baseline setting. The remaining cases exhibit only minor changes, suggesting that grid augmentation primarily benefits bars whose values are harder to judge without explicit reference lines.

% Overall, compared to the baseline, grid-aware enhancement improves prediction quality in most cases across both bubble and bar chart types. A closer inspection reveals that for the bubble chart, in which red crosshairs are closer to the true center of corresponding bubble than blue crosshairs, which indicates all examined data points benefit from grid enhancement relative to the baseline. 
% For the horizontal bar chart, the introduction of grids also yields positive effects for more than half of the data points.
% \wy{How should we find the correspondence between the findings here and what is shown in Figure 5? It is not that clear to me.}
% For donut charts, a closer inspection of the baseline results reveals that the grid-based angular reference also helps reduce ambiguity in angle reading, leading to more accurate initial estimates.

Building on the grid-enhanced predictions, we further examine the effect of iterative visual feedback, which provides localized corrective cues to refine residual errors that remain after grid-aware enhancement.
% Building on the grid-enhanced predictions, we illustrate the effect of iterative visual feedback across different chart types. 
For the bubble chart, as shown in Fig.~\ref{fig:case1} (Bubble), the final predictions obtained after visual feedback are closer to the true centers of most bubbles than the grid-enhanced predictions. Representative examples include bubbles corresponding to \textit{E08} and \textit{D05}, where visual feedback effectively corrects substantial deviations that persist after the grid-based step.
% For the bubble chart, we can see in Fig.~\ref{fig:case-feedback} (bubble), the final predictions from visual feedback get closer to the true center of most bubbles than grid-enhanced prediction. In addition, representative example including the bubbles representing \textit{E08} and \textit{D05}, where feedback further corrects a substantial deviation that remains after grid enhancement.
% In most cases, feedback contributes positively by progressively adjusting predictions toward the ground truth positions. Across chart types, this effect is primarily reflected in the correction of large localization errors that persist after grid enhancement. \wy{Pls check my Chinese comments.}
For the bar chart, iterative visual feedback results in small fluctuations around the grid-enhanced predictions that are occasionally unfavorable in direction but remain bounded within the large baseline error range. As most bars already converge to stable predictions after the grid-based step, visual feedback does not yield additional systematic improvements in this case.
% For the bar chart, predictions from visual feedback present little changes after with the grid-aware enhancement, most bars reach a stable prediction.  
% \wy{where is ``the \textit{C, Group~3} data point'' in Figure 5??? Are you referring to the bar? }
% illustrates a similar pattern: grid enhancement alone was insufficient to correct a large deviation, while after one round of feedback, the prediction returned to a reasonable error range. \wy{How do we know that from Figure 5??}
For the donut chart, visual feedback further refines angle-based proportion inference by addressing several characteristic error patterns. As illustrated in Fig.~\ref{fig:case1} (Donut), feedback mitigates large angular errors (e.g., Sector~B) and reduces confusion between start and end angles in Sectors such as C and E, which can arise under the newly introduced angle-based numerical inference setting.
% For the donut chart, we illustrate three representative changes present at sectors repressenting labbel B, C and E.
% feedback not only mitigates large angular errors (Sector B) but also reduces the likelihood of confusing start and end angles introduced by the new numerical inference with angel estimation.
% , as illustrated by data point \textit{B}.
% \wy{How do we know that from Figure 5?? Where should we look at?}


% We further analyze the effect of iterative feedback across different chart types. In most cases, feedback contributes positively by progressively adjusting predictions toward the ground-truth positions. For example, for the \textit{XX} data point in the bar chart, neither the initial prediction nor grid enhancement alone was sufficient to correct a large deviation; however, after one round of feedback, the prediction returned to a reasonable error range. A similar corrective effect can be observed for data point \textit{E08} in the bubble chart. For donut charts, feedback not only mitigates large angular errors but also reduces the likelihood of confusing start and end angles, as illustrated by data point \textit{XX}.

% At the same time, feedback does not uniformly lead to monotonic error reduction for all data points. For instance, for the data point \textit{C, Group~3} in the bar chart, while feedback corrects a large initial deviation, the second feedback round results in a slightly larger error relative to the first. This observation is consistent with our ablation results, indicating that feedback alone provides limited accuracy gains. Nevertheless, by improving localization reliability, feedback enables subsequent zoom-in refinement to further improve overall prediction accuracy.

% Nevertheless, by improving localization reliability, feedback serves as a critical intermediate step that enables subsequent zoom-in amplification to further enhance overall prediction performance. 

Importantly, a comparison of the point-wise ground truth coordinates, baseline predictions, and \toolName{} predictions reported in Table~\ref{tab:case_results_feed} shows that, despite occasional non-monotonic behavior at intermediate feedback rounds and a small number of non-optimal cases (e.g., in the donut chart), the final predictions produced by \toolName{} yield lower errors than the baseline for most of the examined data points.

% case-feedback table
\begin{table}[htbp]
\centering
\caption{
% Qualitative comparison on a bubble chart, a horizontal bar chart, and a donut chart.
% For each chart type, we report the ground-truth values alongside the baseline
% MLLM predictions and the outputs of \toolName{}.
Qualitative comparison on a bubble chart, a horizontal bar chart, and a donut chart shown in Case~1 (Fig.~\ref{fig:case1}). 
For each chart type, we report the ground truth values alongside the baseline MLLM predictions and the outputs of \toolName{}.
% \pyl{The category order here (b,a,c) conflicts with the predefined sequence (a,b,c) in Section V.A.2 and Fig.5. Please reorder to a,b,c for consistency.}
}
\label{tab:case_results_feed}
\scriptsize
% \small
\setlength{\tabcolsep}{1.8pt}
\renewcommand{\arraystretch}{1.15}

\begin{adjustbox}{max width=0.85\textwidth}
\begin{tabular}{m{2.3cm}<{\centering} *{10}{c}}
\toprule
\multicolumn{10}{c}{\textbf{Bubble}}\\
\midrule
\diagbox[width=2.6cm,height=0.8cm]{\textbf{Setting}}{\textbf{Point Name}} 
   &       & D01   & D02   & E03   & A04   & D05   & D06   & C07   & E08 \\
\midrule
\multirow{2}{*}{Ground Truth}
  & X & 13.22 & 82.65 & 3.90  & 48.22 & 52.20 & 22.95 & 58.90 & 9.77 \\
  & Y &  1.66 & 60.55 & 78.39 & 13.03 & 37.04 & 80.23 & 74.14 & 13.39 \\
\midrule
\multirow{2}{*}{Baseline}
  & X &  9.25 & 79.75 & -2.00 & 43.00 & 48.75 & 20.50 & 65.50 & \textbf{9.25} \\
  & Y & -4.00 & 63.50 & 86.00 & 18.50 & 41.00 & 86.00 & \textbf{74.75} & 18.50 \\
\midrule
\multirow{2}{*}{\textbf{VisHintPrompt}}
  & X & \cellcolor{gray!15}\textbf{13.52} & \cellcolor{gray!15}\textbf{82.45} & 
        \cellcolor{gray!15}\textbf{4.27}  & \cellcolor{gray!15}\textbf{47.98} & 
        \cellcolor{gray!15}\textbf{54.25} & \cellcolor{gray!15}\textbf{24.06} & 
        \cellcolor{gray!15}\textbf{59.23} & \cellcolor{gray!15}10.75 \\
  & Y & \cellcolor{gray!15}\textbf{3.23}  & \cellcolor{gray!15}\textbf{60.82} & 
        \cellcolor{gray!15}\textbf{78.97} & \cellcolor{gray!15}\textbf{11.94} & 
        \cellcolor{gray!15}\textbf{36.98} & \cellcolor{gray!15}\textbf{80.00} & 
        \cellcolor{gray!15}\textbf{74.75} & \cellcolor{gray!15}\textbf{13.47} \\
        
\midrule[1pt]
\multicolumn{10}{c}{\textbf{Horizontal Bar (Group 2)}} \\
\midrule
\textbf{Setting} 
 & A & B & C & D 
  & E & F & G & H & I & J \\
\midrule
Ground Truth    
 & 64.00 & 40.00 & 34.00 & 25.00 
  & 17.00 & 27.00 & 15.00 & 17.00 & 7.00 & 14.00 \\
Baseline        
  & \textbf{64.00} & \textbf{40.00} & 39.00 & \textbf{25.00} 
  & \textbf{17.00} & \textbf{27.00} & \textbf{15.00} & \textbf{17.00} & 14.00 & 15.00 \\
\textbf{VisHintPrompt} 
 & \cellcolor{gray!15}\textbf{64.00} 
    & \cellcolor{gray!15}\textbf{40.00} 
    & \cellcolor{gray!15}\textbf{34.00} 
    & \cellcolor{gray!15}\textbf{25.00} 
    & \cellcolor{gray!15}\textbf{17.00} 
    & \cellcolor{gray!15}\textbf{27.00} 
    & \cellcolor{gray!15}\textbf{15.00} 
    & \cellcolor{gray!15}\textbf{17.00} 
    & \cellcolor{gray!15}\textbf{7.00} 
    & \cellcolor{gray!15}\textbf{13.75} \\
  
\midrule[1pt]
\multicolumn{10}{c}{\textbf{Donut}} \\
\midrule
\textbf{Setting} 
  & &  A & B & C & D
  & E & F & G & H \\
\midrule
Ground Truth    
  & & 0.060 & 0.190 & 0.100 & 0.050 
  & 0.050 & 0.200 & 0.260 & 0.090 \\
Baseline  
  & & 0.083 & 0.160 & 0.120 & \textbf{0.063} 
  & \textbf{0.055} & 0.250 & \textbf{0.250} & 0.120 \\
\textbf{VisHintPrompt} & 
  & \cellcolor{gray!15}\textbf{0.056} & \cellcolor{gray!15}\textbf{0.192} &
    \cellcolor{gray!15}\textbf{0.106} & \cellcolor{gray!15}0.067 &
    \cellcolor{gray!15}0.083 & \cellcolor{gray!15}\textbf{0.200} &
    \cellcolor{gray!15}0.233 & \cellcolor{gray!15}\textbf{0.111} \\
\bottomrule
\end{tabular}
\end{adjustbox}
\end{table}
% Importantly, a comparison of the point-wise ground truth coordinates, baseline predictions, and \toolName{} predictions reported in Table \ref{tab:case_results_feed} 
% % \wy{It should be Table III.}
% shows that, despite non-monotonic behavior at intermediate feedback rounds, the final predictions achieved by \toolName{} are substantially more accurate than the baseline across the examined data points.


% To more concretely illustrate how the proposed scaffolded visual-hint
% prompting strategy operates across heterogeneous chart families, we perform
% a qualitative analysis on three representative chart types: scatter plots
% (Cartesian) and donut and rose charts (polar) with distinct value-encoding
% styles. For each case, we visualize the iterative refinement trajectory
% across the feedback and amplifier stages, using the final feedback
% prediction as the anchor for the initial crop that seeds the first round of
% zoom-in refinement.

\begin{figure*}[tb]
  \centering
  \includegraphics[width=\textwidth,
                   alt={Examples of scatterplot data inference with an MLLM.}]{figs/case1.png}
  \caption{%
    Qualitative examples of \toolName{} on four chart
families: horizontal bar, scatterplot, donut, and rose. Each dashed
red box highlights the iterative zoom-in process guided by the feedback
prediction, illustrating how the model progressively converges to the
correct region in the final refinement component.
  }
  \label{fig:case}
\end{figure*}


% \wy{Please revise Case 2 by checking and addressing similar issues existing in Case 1.}

% case-amp table
\begin{table*}[htbp]
\centering
\caption{
Qualitative comparison on a horizontal bar chart, a scatterplot, a donut, and a rose chart shown in Case~2 (Fig.~\ref{fig:case}).
For each chart type, the table reports the ground truth coordinates of the labeled visual marks in the corresponding charts, together with the baseline MLLM predictions and the outputs of \toolName{}.
}
\label{tab:case_results}
\small
\setlength{\tabcolsep}{3.3pt}
\renewcommand{\arraystretch}{1.25}

\begin{adjustbox}{max width=\textwidth, center}
\begin{tabular}{m{2.3cm}<{\centering} *{12}{m{1.6cm}<{\centering}}}
% \toprule
\specialrule{1.3pt}{0pt}{0pt}
\multicolumn{13}{c}{\textbf{Horizontal Bar Chart}}\\
\midrule
\textbf{Setting} 
& Municipal 
& Cooperative 
& Retail power marketer 
& Investor owned 
& Political subdivision 
& Wholesale power marketer 
& Transmission 
& Community choice aggregator 
& State 
& Municipal marketing authority 
& Behind the meter 
& Federal \\
\midrule
Ground Truth 
& 950 & 853 & 266 & 178 & 131 & 32 & 20 & 19 & 17 & 14 & 15 & 9 \\
Baseline 
& 945 & \textbf{850} & \textbf{270} & 188 & \textbf{130} & 40 & 25 & 25 & 25 & 20 & 25 & 20 \\
\textbf{VisHintPrompt} 
& \cellcolor{gray!15}\textbf{950} 
& \cellcolor{gray!15}\textbf{850} 
& \cellcolor{gray!15}\textbf{270} 
& \cellcolor{gray!15}\textbf{180} 
& \cellcolor{gray!15}\textbf{130} 
& \cellcolor{gray!15}\textbf{31} 
& \cellcolor{gray!15}\textbf{19.5} 
& \cellcolor{gray!15}\textbf{19} 
& \cellcolor{gray!15}\textbf{20} 
& \cellcolor{gray!15}\textbf{12.5} 
& \cellcolor{gray!15}\textbf{17.5} 
& \cellcolor{gray!15}\textbf{10} \\
\bottomrule
\end{tabular}
\end{adjustbox}

% \vspace{1.0em} % 与下方原始表格稍微拉开一点距离

\begin{adjustbox}{max width=\textwidth}
\begin{tabular}{m{2.3cm}<{\centering} *{16}{c}}
% \toprule
\specialrule{0.1pt}{0pt}{0pt}
\multicolumn{16}{c}{\textbf{Scatterplot}}\\
\midrule
\diagbox[width=2.6cm,height=0.8cm]{\textbf{Setting}}{\textbf{Point Name}} &  & GBR & ESP & AUS & PHL & POL & ITA & ARG & CHN & EGY & MEX & THA & SAU & IND & COL & IRN \\
\midrule
\multirow{2}{*}{Ground Truth} 
 & X & 61.78 & 52.08 & 58.29 & 66.99 & 66.63 & 65.01 & 59.55 & 69.51 & 62.48 & 67.34 & 58.85 & 63.68 & 70.48 & 64.62 & 66.54 \\
 & Y & 1779.98 & 1104.30 & 1462.91 & 1220.70 & 1371.76 & 1854.68 & 1616.10 & 1297.37 & 1255.41 & 1806.33 & 1536.34 & 1177.09 & 1209.53 & 1701.17 & 1894.95 \\
\midrule
\multirow{2}{*}{Baseline} 
 & X & 62.00 & \textbf{52.00} & 57.30 & 66.74 & 66.50 & 65.10 & \textbf{59.50} & 68.50 & 62.00 & 67.00 & \textbf{59.01} & 63.10 & 70.97 & 64.50 & 67.00 \\
 & Y & 1740.00 & \textbf{1100.00} & 1475.76 & \textbf{1222.73} & 1333.85 & 1857.60 & \textbf{1619.70} & \textbf{1300.00} & 1278.79 & 1750.00 & 1524.62 & 1140.00 & 1160.00 & \textbf{1700.00} & 1874.24 \\
\midrule
\multirow{2}{*}{\textbf{VisHintPrompt}} 
 & X & \cellcolor{gray!15}\textbf{61.70} & \cellcolor{gray!15}\textbf{52.00} & \cellcolor{gray!15}\textbf{58.39} & \cellcolor{gray!15}\textbf{67.00} & \cellcolor{gray!15}\textbf{66.72} & \cellcolor{gray!15}\textbf{64.92} & \cellcolor{gray!15}59.67 & \cellcolor{gray!15}\textbf{69.50} & \cellcolor{gray!15}\textbf{62.44} & \cellcolor{gray!15}\textbf{67.42} & \cellcolor{gray!15}59.22 & \cellcolor{gray!15}\textbf{63.66} & \cellcolor{gray!15}\textbf{70.47} & \cellcolor{gray!15}\textbf{64.64} & \cellcolor{gray!15}\textbf{66.67} \\
 & Y & \cellcolor{gray!15}\textbf{1772.73} & \cellcolor{gray!15}\textbf{1100.00} & \cellcolor{gray!15}\textbf{1466.09} & \cellcolor{gray!15}1224.24 & \cellcolor{gray!15}\textbf{1377.27} & \cellcolor{gray!15}\textbf{1854.55} & \cellcolor{gray!15}1627.27 & \cellcolor{gray!15}1290.00 & \cellcolor{gray!15}\textbf{1263.64} & \cellcolor{gray!15}\textbf{1800.00} & \cellcolor{gray!15}\textbf{1545.45} & \cellcolor{gray!15}\textbf{1186.36} & \cellcolor{gray!15}\textbf{1211.37} & \cellcolor{gray!15}\textbf{1700.00} & \cellcolor{gray!15}\textbf{1900.00} \\
\midrule[1pt]

% Donut Chart (Left) + Rose Chart (Right)
\multicolumn{8}{c}{\textbf{Donut Chart}} & \multicolumn{8}{c}{\textbf{Rose Chart}}\\
\midrule
\textbf{Setting} & & A & B & C & D & E & F &  & FIFF & LQVB & WMIZ & KEZD & RC & EMVG & TJ & YAP \\
\midrule
Ground Truth & & 0.090 & 0.180 & 0.130 & 0.200 & 0.230 & 0.180 &  & 0.52 & 0.95 & 1.00 & 0.58 & 0.96 & 0.92 & 0.63 & 0.73 \\
Baseline & & 0.167 & 0.167 & 0.167 & \textbf{0.170} & 0.170 & 0.170 &  & 0.40 & 0.80 & \textbf{1.00} & 0.82 & 0.60 & 0.66 & 0.30 & 0.30 \\
\textbf{VisHintPrompt} & & \cellcolor{gray!15}\textbf{0.097} & \cellcolor{gray!15}\textbf{0.175} & \cellcolor{gray!15}\textbf{0.125} & \cellcolor{gray!15}0.167 & \cellcolor{gray!15}\textbf{0.233} & \cellcolor{gray!15}\textbf{0.172} &  & \cellcolor{gray!15}\textbf{0.54} & \cellcolor{gray!15}\textbf{0.91} & \cellcolor{gray!15}0.97 & \cellcolor{gray!15}\textbf{0.50} & \cellcolor{gray!15}\textbf{0.81} & \cellcolor{gray!15}\textbf{0.90} & \cellcolor{gray!15}\textbf{0.59} & \cellcolor{gray!15}\textbf{0.53} \\
\bottomrule
\end{tabular}
\end{adjustbox}
\end{table*}



\subsubsection{\textbf{Case2. Progressive Zoom-in Refinement}}

Four chart types are used to illustrate progressive zoom-in refinement, namely a horizontal bar chart, a scatterplot, a donut chart, and a rose chart.
The refinement crop is generated around the model’s prediction from the previous round. Accordingly, for horizontal bar charts, scatterplots, and rose charts, correctness is indicated when the target mark lies near the geometric center of the cropped region.
For donut charts, correctness is indicated when the target sector is centered in the cropped view, with the left and right sectors arising only from the symmetric angular tolerance added to the previous prediction.

Fig.~\ref{fig:case} summarizes the refinement trajectories by illustrating the iterative evolution of model predictions, while Table~\ref{tab:case_results} reports the corresponding ground truth values together with the baseline and \toolName{} predictions.

Across both Cartesian and polar coordinate systems, the qualitative evidence is consistent with the quantitative findings. 
For all four chart types, refinement starts from the model’s feedback-based prediction in the previous round, and successive zoom-in steps progressively move the prediction closer to the geometric center of the cropped region, even when the initial feedback prediction exhibits large deviations. 
This progressive zoom-in process visibly reduces spatial uncertainty in the qualitative trajectories shown in Fig.~\ref{fig:case}, while the corresponding numerical improvements after fine-grained refinement are reflected in Table~\ref{tab:case_results}. 
These qualitative and quantitative observations demonstrate the robustness, generality, and cross-family applicability of \toolName{} under progressive zoom-in refinement. 


We present a detailed analysis for each chart type as follows:

\begin{itemize}
    \item \textbf{Horizontal Bar Chart.} 
    % As shown in Fig.~\ref{fig:case} (horizontal bar) and the corresponding entries in Table~\ref{tab:case_results}, all bars in the horizontal bar chart exhibit improved prediction accuracy under \toolName{}.
    We first illustrate a challenging horizontal bar chart containing several bars with very short lengths, which pose substantial challenges to the baseline model, as shown in Fig.~\ref{fig:case} and the corresponding entries in Table~\ref{tab:case_results}.
    The progressive zoom-in refinement enables increasingly fine-grained numerical inference, as predictions for all bars progressively move toward the center of the cropped region, including both bars of moderate length and those encoding extremely small values.
    % Notably, this includes bars encoding very small values, which are typically more challenging to estimate accurately.
    Representative examples include the bars corresponding to \textit{Behind the meter}, \textit{Political subdivision}, and \textit{Transmission}, which encode extremely small values that are difficult for the baseline model to estimate accurately, and are progressively corrected through successive zoom-in refinement.    
    This case highlights the importance of progressive zoom-in refinement for fine-grained numerical inference in horizontal bar charts, particularly when estimating values encoded by very short bars.
    % This case illustrates that progressive zoom-in refinement in \toolName{} is particularly effective for improving numerical estimation of short bars and small values, where baseline predictions are more prone to error.
    \item \textbf{Scatterplot.} 
    We use this scatterplot case to illustrate how progressive refinement operates in two-dimensional numerical prediction. 
    As shown in Fig.~\ref{fig:case} (Scatterplot), the predicted scatter points progressively converge toward the center of the cropped region over refinement iterations, yielding effective fine-grained numerical inference that is consistent with the accuracy improvements reported for \toolName{} in Table~\ref{tab:case_results}.
    In particular, points labeled \textit{AUS}, \textit{ITA}, and \textit{IND}, highlighted with red bounding boxes, exhibit more challenging initial deviations, with feedback predictions noticeably offset from the crop center, yet are progressively corrected through successive refinement iterations. 
    This demonstrates the effectiveness of progressive refinement in transforming coarse, error-prone point estimates into stable and fine-grained numerical predictions in two-dimensional space.


    
    \item \textbf{Donut chart.} 
    This donut chart case is used to illustrate how progressive refinement performs in proportion prediction using a novel angle-based numerical prediction approach. 
    As shown in Fig.~\ref{fig:case} (Donut), successive refinement components progressively bring the target sector toward the center of the cropped view. 
    In particular, labels \textit{A}, \textit{C}, and \textit{F} illustrate typical correction patterns: although the initial angular estimates are noticeably misaligned, progressive zoom-in refinement incrementally steers the prediction toward the correct angular boundaries. 
    In the final crops, the target sectors appear symmetrically framed by their neighboring sectors. 
    Overall, this donut chart case illustrates the effectiveness of progressive refinement in transforming coarse angular estimates into stable and fine-grained proportion predictions.

    \item \textbf{Rose chart.} 
    % Rose charts combine angular segmentation with polarly varying magnitudes, making them considerably more challenging than scatterplot or donut charts.
    This rose chart case is used to illustrate how progressive refinement performs under more complex polar structures that combine angular segmentation with radially varying magnitudes, a setting in which baseline predictions exhibit large deviation errors, as shown in Table~\ref{tab:case_results} (Rose). 
    Despite this inherent difficulty, the final predictions after progressive refinement achieve substantial improvements for most data items.      
    As shown in Fig.~\ref{fig:case} (Rose), even after grid-aware enhancement and visual feedback, the initial predictions can remain noticeably offset from the center of the cropped region for challenging examples such as \textit{EMVG}, \textit{TJ}, and \textit{LQVB}, but are progressively corrected toward the crop center over successive refinement iterations.    
    Notably, for example \textit{EMVG}, the cropping window does not shrink monotonically in the second refinement round. When the refined window no longer contains the target sector, the verification mechanism adaptively enlarges the crop to reestablish a valid context, allowing subsequent refinement to proceed from a valid region and recover from severe initial errors.    
    Overall, this rose chart case illustrates the effectiveness of progressive zoom-in refinement in reducing large initial deviations and enabling fine-grained numerical prediction in more complex polar chart structures.


    
\end{itemize}
% \paragraph*{Horizontal Bar Chart (Cartesian)}


\subsection{Ablation Study}

We examine the contribution of each 
elicitation component 
% \wy{Pls use the same descriptions for the same thing across the whole paper! Components or stages???}
in
\toolName{} by conducting ablations on one representative chart type per
coordinate family: horizontal bar, donut and radar charts. 
% All variants use the same backbone model (Gemini-2.0-Flash) and prompt template.
Due to the computational cost of multi-round inference, the ablation study is conducted on a randomly sampled 40\% subset of the evaluation data for each chart type. 
All variants are evaluated using Gemini-2.0-Flash with the same prompt template, and differ only in the activated elicitation components.
We consider the following settings:
\begin{itemize}
    \item \textbf{Baseline} uses a single forward pass of the vanilla model.

    % \item \textbf{Baseline} applies the vanilla model in a single forward pass.

    \item \textbf{w/o Grid Enhancement} removes the initial axis-guided global
    grid. To keep local refinement meaningful, we retain the fine-grained
    grids drawn inside cropped regions for all variants that use local
    amplification. This ablation therefore isolates the additional benefit
    of global axis priors beyond the grids that are intrinsically coupled
    with progressive zoom-in refinement.
    \item \textbf{w/o Visual Feedback} keeps global grid prompting but
    disables prediction overlay iterations. The model observes the gridded
    chart once and directly outputs values, and the progressive zoom-in refinement
    component uses this single global estimate as its starting point.
    \item \textbf{w/o Zoom-in Refinement} retains global grid hint and visual feedback, while performing all inference at the original resolution.
    % \item \textbf{w/o Zoom-in Refinement} retains global grid prompting and
    % visual feedback but removes the crop-zoom-fine-grid stage, so all
    % inference happens at the original global resolution.
    \item \textbf{\toolName{}} combines all three components: Axis-aware
    Grid Enhancement, Iterative Visual Feedback, and Progressive Zoom-in Refinement
    with fine-grained grids in the cropped regions.
\end{itemize}

\begin{table}[t]
\centering
\caption{
Ablation study of major elicitation components.
We report  RNE and RE (lower is better) on one
representative chart type from each coordinate family.
}
\label{tab:ablation}
\small
\resizebox{\columnwidth}{!}{%
    {\renewcommand{\arraystretch}{1.2}%
    \setlength{\tabcolsep}{3pt}%
    \begin{tabular}{@{} >{\centering\arraybackslash}m{3.2cm}cccccc@{}}
    \toprule
    \multirow{2}{*}{\textbf{Method Variant}} 
    & \multicolumn{2}{c}{Horizontal Bar} 
    & \multicolumn{2}{c}{Donut} 
    & \multicolumn{2}{c}{Radar}  
    \\
    \cmidrule(lr){2-3}
    \cmidrule(lr){4-5}
    \cmidrule(lr){6-7}
    & {RNE} & {RE} & {RNE} & {RE} & {RNE} & {RE}  \\
    \cmidrule(lr){1-1}
    \cmidrule(lr){2-3}
    \cmidrule(lr){4-5}
    \cmidrule(lr){6-7}
    % \midrule
    Baseline (Gemini-2.0-Flash) & 5.4 & 14   & 2.88 & 26.86   & 27.1 & 51.7   \\
    w/o Grid Enhancement          & 0.25 & \textbf{0.88}   & --- & ---   & 7.5 & 24.1   \\
    % w/o Grid Prompting          & 0.15 / 0.88   & 2.65 / 31.78   & 7.5 / 14.1   \\
    w/o Visual Feedback         & 0.17 & 1.8   & 2.3 & 26.62   & 7.3 & 22.6  \\
    w/o Zoom-in Refinement    & 4.5 & 23  & 4.04 & 38.9   & 16.4 & 48.0  \\
    \textbf{\toolName{}}  & \cellcolor{gray!15}\textbf{0.15} 
    & \cellcolor{gray!15}0.9 & \cellcolor{gray!15}\textbf{1.99} & \cellcolor{gray!15}\textbf{17.77} & 
    \cellcolor{gray!15}\textbf{7.2} & 
    \cellcolor{gray!15}\textbf{22.3 } \\
    \bottomrule
    \end{tabular}%
    }
}
\end{table}

Table~\ref{tab:ablation} reports RNE and RE (lower is better) for these
variants on the three chart types. 
% \wy{Three or four???}
Overall, every component contributes to the
final performance on RNE: removing any single component degrades results relative
to the full strategy.

% , although the magnitude of the drop differs across components and chart families. 
% The most pronounced effect comes from disabling
% progressive zoom-in refinement. On each type charts, the w/o progressive zoom-in refinement
% variant shows large increases in RNE and RE, indicating that it is the main driver component of precise numerical gains. Without
% this component, the model cannot reliably convert coarse geometric
% understanding into accurate value estimates.
\textbf{Progressive zoom-in refinement is the dominant contributor to numerical accuracy.}
Disabling progressive zoom-in refinement results in the most pronounced performance drop across all chart types. 
For each chart, the w/o progressive zoom-in refinement variant exhibits substantial increases in both RNE and RE, indicating that this component is the primary driver of precise numerical gains. 
Without zoom-in refinement, the model struggles to reliably translate coarse geometric understanding into accurate value estimates.

% \textbf{Grid-aware enhancement provides complementary but geometry-dependent benefits.}
\textbf{Axis-aware grid enhancement provides complementary benefits with varying strength across chart geometries.}
Relative to the full \toolName{}, disabling grid prompting leads to overall performance degradation, indicating the effectiveness of the grid component. 
On horizontal bar charts, removing the grid results in higher RNE but a slightly lower RE, suggesting that while grid prompting may not uniformly improve all error metrics in simple Cartesian layouts, it still plays a role in stabilizing range-normalized errors. 
On radar charts, disabling grid prompting leads to more pronounced increases in both RNE and RE. 
This suggests that as chart geometry shifts from simple and intuitive Cartesian layouts to more complex polar structures, numerical inference becomes more challenging, making explicit grid-based axis priors increasingly critical.

\textbf{Iterative visual feedback consistently improves numerical inference across chart types.}
Relative to the full \toolName{}, disabling visual feedback leads to clear error increases on all three chart types, demonstrating that the visual feedback component contributes meaningfully to the overall numerical inference process. 
The degradation is particularly pronounced on donut charts, where numerical values are inferred through a novel angle-based proportion prediction formulation, highlighting the importance of visual feedback for stabilizing angular estimation in this setting.


 
The three elicitation components provide complementary and synergistic benefits.
Taken together, these ablations demonstrate that grid-aware enhancement, visual feedback, and progressive zoom-in refinement each play distinct yet complementary roles. 
Grid prompting and visual feedback guide coarse geometric reasoning and stabilize the entry point into the refinement pipeline, while zoom-in refinement is essential for converting these coarse cues into accurate numerical predictions. 
As a result, the full \toolName{} achieves the strongest overall performance across both Cartesian and polar coordinate families, with the most substantial gains observed when all three components are jointly enabled.











\section{Discussion}
This work demonstrates that explicitly structured visual scaffolding, through axis-aware grids, visual feedback, and progressive zoom-in refinement, substantially improves numerical inference from charts in MLLMs. 
Beyond empirical accuracy gains, our findings shed light on how current models internalize chart geometry, why different elicitation components contribute at distinct levels of abstraction, and how structured visual guidance can support more reliable visual reasoning beyond standard chart settings. 
% We discuss these implications below.

\subsection{Insights and Broader Implications}
% \paragraph{Differential priors in MLLMs for chart geometry}
\subsubsection{\textbf{Uneven Geometric Familiarity in MLLMs}}
% \wy{what are differential priors here?}
% Our results indicate that MLLMs exhibit a structured hierarchy of difficulty across chart types that reflects underlying geometric properties.
Our results indicate that MLLMs differ systematically in their numerical inference performance across chart types. 
Models show more inherent familiarity with chart types where values are mapped to direct and perceptually salient spatial positions, consistent with dominant visual and linguistic conventions in natural image–text data. 
In contrast, chart types that rely on less canonical encodings, including mappings along non-dominant axes as well as area- or angle-based representations, exhibit weaker geometric familiarity. 
These encodings are associated with higher baseline errors and larger performance gains under \toolName{}. 
Overall, these findings suggest that numerical inference in MLLMs is shaped by an interaction between the complexity of value-to-geometry mappings and biases in the model’s training data exposure.




\subsubsection{\textbf{Distinct Roles of Visual Scaffolding Components}}
Our findings suggest that effective numerical inference in MLLMs requires a clear division of labor across different forms of visual scaffolding. 
Global grid cues and visual feedback primarily operate at a coarse level by making chart structure and approximate value ranges explicit, which helps stabilize initial localization. 
Progressive zoom-in refinement, in contrast, provides the fine-grained spatial resolution needed to translate this coarse localization into accurate numerical estimates. 
This separation of roles explains why global structural guidance alone can identify relevant regions but lacks numerical precision, while local refinement alone is prone to unstable localization, making their combination essential for reliable numerical reasoning.



\subsubsection{\textbf{Externalizing Structured Priors as Visual Scaffolds}}
The design of \toolName{} illustrates a broader methodological principle for structured visual reasoning tasks. 
Many visual representations, particularly charts, engineered schematics, and other diagrammatic artifacts, rely on explicit structural conventions that humans readily exploit but that MLLMs do not reliably infer from raw pixels alone. 
By rendering this structure directly in the input through grids, overlays, and localized refinements, the prompting pipeline functions as an inference-time scaffold that constrains reasoning within a well-defined geometric frame. 
This principle is likely to generalize to domains such as circuit diagrams, process flows, and technical schematics, where consistent visual grammars exist but are underrepresented in model training data. 
Externalizing such structure provides a lightweight and domain-adaptable mechanism for enhancing multimodal reasoning without modifying model parameters.



\subsection{Limitations and Future Work}

\subsubsection{\textbf{Dependence on Mark Count and Labeling Density}}
Our strategy assumes that visual elements remain distinguishable under
progressive crops. Charts with extremely dense marks or overlapping
labels challenge this assumption. In such cases, the refinement component may
struggle to preserve semantic correspondence between the cropped region
and the original chart. A potential solution is a two-step labeling
pipeline: the model first proposes a coarse identity tag for each mark
(e.g., region, item, or category), and the refinement procedure then
operates relative to these inferred identities. Preliminary experiments
on scatterplots with off-center labels suggest that this approach can
help maintain stable grounding even under dense or cluttered layouts.


\subsubsection{\textbf{Inference-time Efficiency}}
The visual hint prompting strategy introduces additional inference-time overhead due to multi-round prompting, iterative visual feedback and progressive zoom-in refinement. 
While this design is essential for improving numerical precision, it may limit scalability in latency-sensitive or large-scale deployment settings.


\subsubsection{\textbf{Future Work}}
Several promising directions for future research are worth exploring. 
First, integrating \toolName{} into end-to-end chart understanding systems would enable evaluation on large-scale, naturally occurring datasets and more complex analytical tasks. 
Second, extending structured visual scaffolding to other schematic domains, such as technical diagrams or process flows, could further assess the generality of axis externalization and coarse-to-fine refinement. 
Finally, future work may explore hybrid prompting strategies that incorporate adaptive or learned zoom policies within an inference-time visual scaffolding framework, aiming to better balance numerical accuracy and inference efficiency.



\section{Conclusion}
We introduce \toolName{}, a novel visual hint prompting strategy designed to unleash the potential of MLLMs in chart data extraction tasks across diverse chart geometries and coordinate systems, including both Cartesian and polar representations.
Unlike traditional approaches that depend on task-specific training or fine-tuning, 
\toolName{} integrates chain-of-thought prompting with visual-hint-based multimodal strategies, enabling the model to perform numerical inference directly over grid-enhanced chart images. The strategy iteratively refines predictions through a visual feedback mechanism and leverages progressive zoom-in refinement to capture fine-grained visual details, thereby enhancing the model’s perception and interpretation of target data objects. Experimental results demonstrate that \toolName{} constantly improves both interpretability and accuracy of numerical extraction in challenging scenarios, such as small-scale graphical elements. 



% \section*{Acknowledgments}
% This should be a simple paragraph before the References to thank those individuals and institutions who have supported your work on this article.



% {\appendix[Proof of the Zonklar Equations]
% Use $\backslash${\tt{appendix}} if you have a single appendix:
% Do not use $\backslash${\tt{section}} anymore after $\backslash${\tt{appendix}}, only $\backslash${\tt{section*}}.
% If you have multiple appendixes use $\backslash${\tt{appendices}} then use $\backslash${\tt{section}} to start each appendix.
% You must declare a $\backslash${\tt{section}} before using any $\backslash${\tt{subsection}} or using $\backslash${\tt{label}} ($\backslash${\tt{appendices}} by itself
%  starts a section numbered zero.)}



%{\appendices
%\section*{Proof of the First Zonklar Equation}
%Appendix one text goes here.
% You can choose not to have a title for an appendix if you want by leaving the argument blank
%\section*{Proof of the Second Zonklar Equation}
%Appendix two text goes here.}



% \section{References Section}
% You can use a bibliography generated by BibTeX as a .bbl file.
%  BibTeX documentation can be easily obtained at:
%  http://mirror.ctan.org/biblio/bibtex/contrib/doc/
%  The IEEEtran BibTeX style support page is:
%  http://www.michaelshell.org/tex/ieeetran/bibtex/
 

% 参考文献
\bibliographystyle{IEEEtran}
\bibliography{template}


% \newpage

% \section{Biography Section}
% If you have an EPS/PDF photo (graphicx package needed), extra braces are
%  needed around the contents of the optional argument to biography to prevent
%  the LaTeX parser from getting confused when it sees the complicated
%  $\backslash${\tt{includegraphics}} command within an optional argument. (You can create
%  your own custom macro containing the $\backslash${\tt{includegraphics}} command to make things
%  simpler here.)
 


\begin{IEEEbiography}[{\includegraphics[width=1in,height=1.25in,clip,keepaspectratio]{photos/ZHENG FENGLING_photo.jpg}}]{Fengling Zheng}
is currently a PhD in the School of Computer Science, Hangzhou Dianzi University, and a researcher in Big Data Visualization and Human Computer Collaborative Intelligent Laboratory. Her research interests include interactive visualization, visual analytics and knowledge graph.
\end{IEEEbiography}
\vspace{-1.0\baselineskip}

\begin{IEEEbiography}[{\includegraphics[width=1in,height=1.25in,clip,keepaspectratio]{photos/pengyongle.png}}]{Yongle Peng}
is currently a senior undergraduate student majoring in Digital Media Technology at the School of Humanities, Arts and Digital Media, Hangzhou Dianzi University. His research interests include multimodal large language models, computer vision, and visual analytics.
\end{IEEEbiography}
\vspace{-1.0\baselineskip}

\begin{IEEEbiography}[{\includegraphics[width=1in,height=1.25in,clip,keepaspectratio]{photos/rongzi.png}}]{Zi Rong}
is currently a Master’s student in the School of Humanities, Arts and Digital Media, Hangzhou Dianzi University, and a researcher in Big Data Visualization and Human Computer Collaborative Intelligent Laboratory. Her research interests include chart understanding, data visualization, and human–computer interaction.
\end{IEEEbiography}
\vspace{-1.0\baselineskip}

\begin{IEEEbiography}[{\includegraphics[width=1in,height=1.25in,clip,keepaspectratio]{photos/caichengyun.jpg}}]{Chenyun Cai}
is currently an undergraduate student in Digital Media Technology at Hangzhou Dianzi University. Concurrently, he is a Research Assistant at the Lab of Big Data Visualization and Human-Computer Collaborative Intelligence, where his research focuses on knowledge graphs and interactive visualization.
\end{IEEEbiography}

\begin{IEEEbiography}[{\includegraphics[width=1in,height=1.25in,clip,keepaspectratio]{photos/heyumeng.jpg}}]{Yumeng He}
is a Master’s student in Computer Science at the University of Southern California, advised by Dr. Jernej Barbič. Her research interests lie at the intersection of computer vision, computer graphics, and robotics, with a focus on image and video understanding, 3D and 3DGS generation, physics-based simulation, real-to-sim pipelines, and embodied policy learning.
\end{IEEEbiography}

\begin{IEEEbiography}[{\includegraphics[width=1in,height=1.25in,clip,keepaspectratio]{photos/qiandekun.jpg}}]{Dekun Qian}
is currently a Master student in the School of Humanities, Arts and Digital Media, Hangzhou Dianzi University, and a researcher in Big Data Visualization and Human Computer Collaborative Intelligent Laboratory. His research interests include human-computer interaction and digital media technology.
\end{IEEEbiography}


\begin{IEEEbiography}[{\includegraphics[width=1in,height=1.25in,clip,keepaspectratio]{photos/zhouzhig.png}}]{Zhiguang Zhou}
is currently a professor in School of Humanities, Arts and Digital Media, and serves as the dean of Digital Media Technology Research Institute at Hangzhou Dianzi University. His research interests include data visualization, visual analytics and knowledge graph mining. He received his Ph.D. in Computer Science from the state key Laboratory of CAD\&CG in Zhejiang University. 
\end{IEEEbiography}

\begin{IEEEbiography}[{\includegraphics[width=1in,height=1.25in,clip,keepaspectratio]{photos/wangyigang.jpg}}]{Yigang Wang} 
received the MS and PhD degrees in applied mathematics from Zhejiang University, Hangzhou, China. He is currently a professor with the School of Media and Design, Hangzhou Dianzi University, Hangzhou, China. His interests include image processing, computer vision, pattern recognition, and computer graphics.
\end{IEEEbiography}


\begin{IEEEbiography}[{\includegraphics[width=1.0in,height=1.25in,clip,keepaspectratio]{photos/wangyong.png}}]{Yong Wang}
is currently an assistant professor in the College of Computing and Data Science, Nanyang Technological University. Before that, he worked as an assistant professor at Singapore Management University from 2020 to 2024. His research interests include information visualization, visual analytics and human-AI collaboration, with an emphasis on their application to FinTech, quantum computing and online learning. He obtained his Ph.D. in Computer Science from Hong Kong University of Science and Technology. He received his B.E. and M.E. from Harbin Institute of Technology and Huazhong University of Science and Technology, respectively. For more details, please refer to \url{http://yong-wang.org}.
\end{IEEEbiography}

\vfill

\end{document}


